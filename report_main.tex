\documentclass[journal,compsoc]{./IEEEtran}

\usepackage{url}
\usepackage{amsmath}
\usepackage{amsfonts}
\usepackage{graphicx}
\usepackage[utf8]{inputenc}
\usepackage{subcaption}

\usepackage{tabularx}
\usepackage{booktabs,fancyhdr}
%\usepackage{makecell}

\usepackage{nameref}

\usepackage{xcolor}
\usepackage{soulutf8}
% These are for showing the changes
% \newcommand{\changeRemove}[1]{{
% 	\color{white}\sethlcolor{red}\hl{#1}
% }}
% \newcommand{\changeAdd}[1]{
% 	\sethlcolor{green}\hl{#1}
% }
% These are for compiling with only the new changes
\newcommand{\changeRemove}[1]{}
\newcommand{\changeAdd}[1]{#1}


%\usepackage[swedish]{babel}
\usepackage[backend=bibtex, citestyle=ieee, bibstyle=ieee]{biblatex}
\addbibresource{ref.bib}

\begin{document}

\title{Refugees Welcome Housing \\Solutions for a fragmented set of services}

\author{
	\changeAdd{Group 9}\\
	Hannes~Rabo,~
	~Anna~Zakipour,
    ~Julius~Celik,
    ~Ahmad~Jabali,
    ~and~Goran~Ibrahim\\
    KTH - Royal Institue of Technology, Stockholm
    \thanks{Supervisor at KTH - Miriam Börjesson Rivera}
    \thanks{Supervisor from Refugees Welcome Housing Sweden - Kajsa Sörman}}


% The paper headers
\markboth{Sustainable Development, ICT and Innovation (AG1815) - 2018 - KTH}%
{}

\IEEEtitleabstractindextext{%
\begin{abstract}
This report studies the internal and external communication and workflow in Refugees Welcome Home Sweden and asses them according to multiple relevant factors. From this information multiple alternatives that could enable a more efficient and intuitive workflow were created. The conclusion was that most available tools were too expensive for an organization with low revenue. The best option amongst the investigated, would be to use G Suite for non profit organizations temporarily and strive to upgrade the current system (SuiteCRM) to better fit the internal workflow. An increased productivity would further extend the positive effects from their work with important sustainability factors in Swedish society.

% In this report we study the internal communication and workflow in Refugees Welcome Sweden and assess it according to a number of factors. From this information we created multiple alternatives that could make the workflow more efficient and intuitive. We concluded that most tools on the market are too expensive for an organization with low revenue and that the best option would be to use G Suite for non profit organizations temporarily and strive to upgrade the current system (SuiteCRM) to better fit the internal workflow. An increased productivity would further extend the positive effects from their work with important sustainability factors in Swedish cities.
\end{abstract}

\begin{IEEEkeywords}
 Communication Systems, CRM, Sustainable Development, Sharing economy, Social Sustainability, SDG.
\end{IEEEkeywords}
}

% make the title area
\maketitle

% Describe the background to the problem and make a clear problem formulation. Make
% sure the reader understands the context and why the question is interesting from both
% an ICT and sustainability perspective.
% The sustainability aspects must be clearly evident from the problem formulation.

% Problem description: Which problem does the project attempt to solve, why is this in-
% interesting from a sustainability perspective?

\section{Introduction}
\IEEEPARstart{R}{efugees Welcome Housing Sweden} is a non profit organization with the aim of connecting landlords in Sweden with refugees by matching them on a personal basis \cite{Welcome}. If a potential landlord is interested in renting an apartment, letting or subletting their apartment they can easily visit the web page that is run by the organization and fill out their contact information. One key point, and the main purpose of RWHS (\textit{Refugees Welcome Housing Sweden}), is to provide refugees with housing that is cheap enough for them to afford, by providing a personal relationship between them and the landlord. 

This processes of finding and matching landlords and tenants is in the current state not as optimal as it could be. The main issues are related to technical problems with handling the complex systems that are used in the organization. Volunteers use multiple different services to cover their digital needs which is undesirable. A simpler, more complete and integrated system that is easy to understand and get started with, even for non technical users is necessary to enable greater accessibility for everyone \changeAdd{working in the organization}. A solution to this problem would increase the productivity and efficiency for the organization as a whole. An updated system would need to be able to handle personal and sensitive data from refugees as well as landlords in a way that is both easier and legal according to current regulations. It is not necessary for all parts to be integrated, however ease of use is a primary objective and a fragmented system was identified as a primary hindrance for usability.

In this report the current state of the organization were assessed, and the the needs for the different parts of the daily operation were identified. Based on this a proposal for alternative systems were created, that could help improve the workflow and efficiency without compromising security and the strict economic boundaries that were connected to this case.

% In this report we try to assess the current state of the organization and identify needs for different parts of the daily operation. Based on this we will create a proposal for alternative systems that could help improve the workflow and efficiency without compromising security and the strict economic boundaries that are connected to this case.

\subsection{Problem}
\label{sec:problem-definition}
After an initial study of the organization, three key problem points could be identified which this report investigated.
\begin{enumerate}
\item \textit{Over separation of internal and external information sharing, caused by a fragmented system of services.} Easy communication, both internal and external, is needed to organize workers and inform them about procedures and regulations. The current solution stores and communicates information across a multitude of platforms which are hard to synchronize. 
\item \textit{Usage problems for new as well as more experienced volunteers.} The volunteers are required to learn a multitude of services, just for get started. This time is lost to technical details instead of starting on the important work. It disconnects the volunteers from working at the major issue. Beyond being an extra time waste, it creates a hindrance for volunteers and decreases morale within the organization as the current procedure is way too inefficient.
\item \textit{Secure handling of personal data.} RWHS handles personal and sensitive data from refugees, as well as landlords. Substantial parts of this information is lawfully required to be securely kept, which has to be considered when proposing a digital solution.
\end{enumerate}

\subsection{Sustainability}
\label{sec:introduction:sustainability}
RWHS as an organization is in the current state trying to tackle a problem, that even with large resources, would be complex and hard to solve in a good way. At the same time their organization is completely dependent on the willingness and passion from volunteers as well as donations to be able to continue operating. Whilst RWHS are in a limited situation economically, they are providing newly arrived \changeRemove{families}\changeAdd{refugees} with more personal relationships and better treatments than elsewhere. This also means that the \changeRemove{families}\changeAdd{refugees} have a larger chance of successfully establishing themselves in Sweden. Not only does this put the families in a better place for building a new and successful life in Sweden, yet it also eases their integration into the Swedish society which is important for creating an accepting and socially sustainable environment. This type of work improves upon the condition of UN's SDGs
(\textit{Sustainable Development Goals)} \cite{UnitedNations} such as: 
\begin{itemize}
\item Goal 10 - Reduced Inequalities
\item Goal 11 - Sustainable cities and communities 
\item Goal 12 - Responsible consumption and production
 \item Goal 16 - Peace, justice and strong institutions
\end{itemize}
More information about the goals and their connections to RWHS are found in section \ref{sec:background:sdg} and \ref{sec:discussion-sdg} (\textit{\nameref{sec:background:sdg}}).

Because of the economical limitations of a non profit organization it is sometimes difficult to acquire adequate knowledge and funds to improve IT systems, which is a struggle that commercial companies do not experience to the same extent. If a replacement of the current system for a better alternative is possible it would contribute to an increase of time efficiency and decrease the risk of volunteers dropping of. This means that more energy could be spent on the tasks that actually matters for the volunteers. This implies that our work as an unbiased source of information for the organization indirectly helps to achieve a more sustainable society. 

\section{Theory and background}
\label{sec:background}
In this section we discuss the different tools and background that is needed to understand the current and possible future solutions in more detail. We first develop on the concept of PUL and GDPR with restrictions in handling personal data (section \ref{sec:pul} and \ref{sec:gdpr}). After that we briefly introduce the tools that are used within the organization today (section \ref{sec:current-tools}).

\subsection{PUL / PDA}
\label{sec:pul}
The Personal Data Act (PUL) is a Swedish law that exists to protect people and their personal integrity.  In general the law covers how personal documents should be handled to be safe as well as prevent unauthorized people from accessing and misusing them.  One important aspect here is that information should not be available to people that don't need it.
% It protects all records that counts as personal data and limits how they can be handled. Redundant (Julius)
Personal data or information is a wide term here that includes everything which could potentially be used to identify an individual \cite{Datainspektionen2015}. The full definition from Datainspektionen in Swedish is available in appendix \ref{app:personal-information}.

According to the definition we call anyone using a cloud service for processing personal data \textit{the controller} of the personal data, even if the processing is carried out by a cloud service provider or its subcontractors. PUL forces any controller of personal data, who also makes use of cloud services, to ensure that a new third party cloud provider does not violate the law and that the following criteria are met \cite{DATAINSPEKTIONEN} :
\begin{enumerate}
\item Ensure that there are no risks that the personal data may be used for other purposes than the original one,
\item Ensure that the personal data saved in the cloud service can’t be available to a third country, i.e in a country outside of EU,
% But if stored in a country outside, it is supporting the Personal Data Act.
\item Assess what security measures has to be taken in order to protect the personal data that is processed,
\item Ensure that a processor agreement is drawn up with the cloud provider, and also
\item Consider other legislation such as confidentiality legislation.
\end{enumerate}

\subsection{GDPR}
\label{sec:gdpr}
From the 25th of may, a new law called GDPR (\textit{General Data Protection Regulation}), will replace PUL. The GDPR has been constituted by the EU (\textit{European Union}) and will replace data protection laws in all countries where GDPR is active. The current legislation for many countries  were enacted before cloud services and Internet where widespread, which is a reason that the previous laws are a bit outdated. The GDPR is supposed to strengthen data security and improve trust in the digital era. At the same time it will benefit business by unifying the laws on the European market which simplifies legal processes.

It is relevant to take into consideration that when the laws are changed, GDPR will apply to any company that are dealing with data that belongs to residents in the EU. This means that it will be possible for companies to provide services for handling personal information, even if they host their data outside of EU. It introduces better possibilities for transferring data and makes it economically possible to create compliant systems because of the reduced differences between different countries \cite{Google}. Another big change is that if you fail to follow GDPR, the penalties can be as large as 20 000 000 Euro or 4\% of the conglomerate global revenue which makes it key to follow \cite{MULTISOFT}. This can be compared to PUL where penalties have been both less common and severe.

Work on the GDPR started 2016, however the EU agreed with the final text that all organizations and businesses has to comply with the GDPR by the 25th of May 2018. Similar to PUL, both controllers and processors of data need to comply with the GDPR. The controller would in this case be RWHS, whereas the processor would be the company actually handling the data processing. \cite{EuropeanComission2016}

\subsection{Tools}
\label{sec:current-tools}
The organization is as previously mentioned using a multitude of different tools to fulfill their different needs. In this section we will briefly introduce some of the services that are used today: SuiteCRM, Trello, G Suite, Slack, as well as a few other daily communication tools. Some prospective candidate tools are also presented as Microsoft 365 Enterprise.

\subsubsection{SuiteCRM}
SuiteCRM is a software fork of the CRM (\textit{customer relationship management}) system SugarCRM. SuiteCRM was released October 2013 and is a free open source alternative to the more complex SugarCRM. RWHS are using SuiteCRM for tracking all relationships and contacts with landlords and tenants. It is also used as a tool for the volunteers to keep track of the states of the different application processes. \cite{CommunitySuiteCRM2014AboutDocumentation}

\subsubsection{Trello}
\label{sec:background:trello}
Trello is an on-line tool for managing projects and progress. It can be shared among a group to organize projects in a task based form \cite{Trello}. RWHS are using Trello to assign tasks to volunteers and control the progress of matching hosts with tenants. According to their own documentation, Trello is not following the laws (as presented in section \ref{sec:pul} and \ref{sec:gdpr}) regarding secure handling of personal data \cite{Trelloa}. 

\subsubsection{G Suite (Gmail, Google Drive, and Google Hangouts)}
\label{sec:background:g-suite}
G Suite is a complete office suite from Google that includes services such as word processors, spreadsheets, emails, forms with data collection as well as file storage. The tools are available for individuals and organizations. Today this is the primary office suite for RWHS and all previously mentioned features are used on a regular basis. \cite{GoogleAnvandarvillkorSuite}

Google has made a clear statement that they will follow GDPR, which opens up for the possibility to use G Suite, both in the free form but also the enterprise version which is available for free to non-profit organizations \cite{Googlea}. Google also commits to help their customers with their GDPR compliance journey by providing security and privacy protections built into all their services and even offers to sign a contract ensuring this. RWHS has already applied for the non-profit program which they got granted.

\subsubsection{Slack}
Slack is a work communication platform that provides teams with quick and organized text based messages. Slack provides a single chat room for the entire organization where individual chat as well as group chat is provided. RWHS are using Slack for communication on a local as well as national level. \cite{SlackAboutSlack}

\subsubsection{Other communication tools}
RWHS has used Slack as their primary daily communication tool however volunteers often choose another communication tool based on previous experience and preference. Both Facebook, Whatsapp, and Hangouts are regularly used by different members.

\subsubsection{Microsoft Office 365 Enterprise}
\label{sec:background:microsoft-office}
Microsoft has a complete office package that includes all basic tools for communication and recording information. These includes things such as email clients, word processors and spread sheets. It also includes the tool Microsoft Tasks which could replace Trello. \cite{MicrosoftMicrosoftNonprofits}

\subsection{Sustainability}
\label{sec:background:sustainability}
This chapter introduces relevant sustainability aspects, that will later be discussed in section \ref{sec:discussion}. Section \ref{sec:background:social-sustainability} introduces social sustainability aspects that are relevant to RWHS's work, section \ref{sec:background:sharing-economy} introduces the concept of sharing economy, and section \ref{sec:background:sdg} explains the UN's SDGs whilst focusing on the four most relevant goals.

\subsubsection{Social sustainability}
\label{sec:background:social-sustainability}
The definition of social sustainability has been widely discussed. One definition developed by Social Life is  \textit{"a process for creating sustainable, successful places that promote wellbeing, by understanding what people need from the places they live and work. Social sustainability combines design of the physical realm with design of the social world – infrastructure to support social and cultural life, social amenities, systems for citizen engagement and space for people and places to evolve."} \cite{Woodcraf}

A component of social sustainability that Social Life presents is bridge sustainability, which they describe as changing behavior to achieve bio-physical environmental goals. \cite{Vallance2011}

\subsubsection{Sharing Economy}
\label{sec:background:sharing-economy}
Sharing Economy is an economic term used to describe a system in which assets or services are shared between private individuals, either for free or for a fee. \cite{Albinsson2018TheEconomy}

%% We already said that these to sections discuss how we indirectly affect
%% sustainability factors. It is redundant information to add this again.
% If RWHS implements one of the provided solutions this project will have an indirect effect on the \textit{Sharing Economy} perspective. By making the required tools and the work flow more simple, the volunteers can work more effective and as a result match more landlords with refugees.

%% Yes, as previously mentioned... We don't need to say this again.
% As mentioned the solutions this project offers  only have an indirect effect on this aspect and there is no guarantee that more landlords and refugees will be matched based only on providing more effective digital tools for RWHS.

\subsubsection{Sustainable Development Goals}
\label{sec:background:sdg}
UN has a collection of 17 goals for sustainability. This report focuses on four goals in more depth.

\textit{Goal 10 - Reduced Inequalities} aims to reduce social, economic and political inequalities within and among countries, regardless of age, sex, disability, race, ethnicity, origin, religion or economic status. 
\cite{UnitedNationsb}

\textit{Goal 11 - Sustainable cities and communities} is about making cities and human settlements safe, inclusive, sustainable and resilient. It is relevant to consider that Stockholm has many young adults and there is a current housing crisis. \cite{Sheiban2002Den1800-talet} \cite{UnitedNationsb}

\textit{Goal 12 - Responsible consumption and production} is about forming long term patterns that will be integrated with national and sectoral plans, companies and consumer behavior. \cite{UnitedNationsGoalPlatform}

\textit{Goal 16 - Peace, justice and strong institutions} described by UN: \textit{"Promote peaceful and inclusive societies for sustainable development, provide access to justice for all and build effective, accountable and inclusive institutions at all levels"}. This goal also includes working against physical and sexual abuse of individuals. \cite{UnitedNationGoalPlatform} 
 



% Describe a clear aim/purpose of the project. Make sure it is well motivated by the problem formulation
% You may divide the purpose into an overall purpose (your long term contribution in a  larger perspective) and more concrete goals (what you intend to do/deliver in order to
% reach the overall purpose).
% 8(19)
% Purpose/goal: What are you going to do, what do you want to achieve?

\section{Purpose}
The purpose of this report is to investigate the current situation and propose one or more alternative solutions for communication within the organization. The solution should be based on the background data collected while studying the problems as previously defined in section \ref{sec:problem-definition} (\textit{\nameref{sec:problem-definition}}). The solution should include a list of alternative services to use, with their respective pros and cons. It should also answer the following questions which are based on the problem definition.
\begin{enumerate}
\item Could RWHS improve their methods of working with digital services, for sharing, documenting and communicating information externally and internally by switching tools?
\item Would this alternative solution be able to provide the same quality of service at the same or only slightly greater cost than what is used today?
\item Would a new solution significantly speed up the learning rate and accessibility of information in the organization for all relevant parties?
\item Does the regulations regarding personal and sensitive data affect the choice of a future system?
\end{enumerate}
Furthermore the proposed alternative solutions has to fit the budgetary restrictions of RWHS, as well as comply with Swedish law (PUL/GDPR). The goal is that after the publication of this report, RWHS will be able to apply one of the proposed solutions to their organization. Upon successful implementation they should be able to observe increased efficiency because of a better digital solution. That would in turn hopefully ensure that the long term goals could be reached which includes fewer volunteers dropping of, and more landlords handled in a more efficient way. By optimizing the efficiency of RWHS, this project hopes to contribute to the project undertaken by them which in turn would contribute to  the same goals as the organization is working towards.









\section{Method}
\label{sec:method}
In this section we present the method used to evaluate internal communication channels in the organization and later develop proposals that could improve the current situation. The project was divided into three main parts before the solution could be evaluated togher with RWHS. These areas where interviews (section \ref{sec:method:interviews}), background research of current situation and tools (section \ref{sec:method:research}), and lastly system prototyping (section \ref{sec:method:prototyping}).

\subsection{Interviews}
\label{sec:method:interviews}
Interviews were conducted with active members of the organization to better understand the problem, as well as what was expected from a proposed solution. The goal of the interviews were to collect as much qualitative data as possible. One of the complaints from RWHS was that different parts of the teams were using different tools or found them difficult to understand. Because of this reason it was important to understand all the problems that the actual users had in more detail than what could be provided by a simple poll. At the beginning of the project, Kajsa Sörman introduced the current situation of the organization as a whole. This was then used as the basis for all work performed later on. From this information we decided to continue our work with the following three volunteers:
\begin{enumerate}
\item Fredric Landqvist (Tech- team and head of the organization)
\item Malin Averstad Ryd (Local coordinator of Stockholm)
\item Sara Hadfy Högström (Project Manager in Malmö)
\end{enumerate}
As all these people had different roles in the organization we used different approaches and focused on their respective roles to get maximum coverage and diversity from our answers. This meant that the questions used differed to some degree which we present in the sections below (sections \ref{sec:method:interviews:fredric}, \ref{sec:method:interviews:henrik}, and \ref{sec:method:interviews:malinandsara}).

Besides investigating the organization we simultaneously worked  with companies that could provide alternative tools to cover the needs of RWHS. This meant that we also \changeRemove{had to talk}\changeAdd{talked} to representatives such as Henrik Resare from Easit, a company that sells custom CRM systems.

\subsubsection{Fredric Landqvist}
\label{sec:method:interviews:fredric}
The \changeAdd{purpose of the} interview with Fredric Landqvist (head of the organization and tech team in Sweden) was held to \changeRemove{give}\changeAdd{get} greater insight into the CRM system RWHS used. This provided information of how the system worked, main purposes with the system and reasons behind the technical choices.

\subsubsection{Henrik Resare}
\label{sec:method:interviews:henrik}
The interview with Henrik Resare was held to research another CRM option than SuiteCRM. The purpose of this interview was to get a clearer view of the process Easit sets up for a new cooperation, and also if it was possible for Easit to accommodate the primary demands of RWHS.

\subsubsection{Malin Averstad Ryd and Sara Hadfy Högström}
\label{sec:method:interviews:malinandsara}
To get a better understanding of how the current tools were used by the volunteers Malin and Sara were interviewed. Both of them worked closely with local volunteers in their respective parts of the country and could give insight into their daily workflow. As the interviews had the same purpose, the same primary questions were used as a starting point:
\begin{enumerate}
\item What services are you currently using in your team?
\item For each service: what do you use it for? What is the workflow?
\item What are your most important tools for the work? Prioritize them.
\item What are your main annoyances in the current setup.
\item What would you remove if you could?
\item Do you have any suggestions for tools to try out instead?
\item Do you use SuiteCRM. What do you think about it?
\end{enumerate}

\subsection{Research}
\label{sec:method:research}
In parallel with interviews with the current users, we conducted research of available solutions. The purpose of this was to achieve an overview of the market for the types of systems that could provide at least the same features that RWHS are using today. The three primary research topics/questions that we collected information about were:
\begin{enumerate}
\item How similar organizations satisfy their digital needs?
\item What is the viability of creating a new custom solution within the budget requirements? 
\item What are some potential alternatives for tools that are not functioning efficiently?
\end{enumerate}
For each of the tools researched we assessed them according to the following categories
\begin{enumerate}
\item How does the jurisdiction regarding secure data handling (PUL/GDPR) limit the usage?
\item Would a switch to this tools create any significant improvement?
\item Would the gain, in efficiency as well as happiness among volunteers, stand in proportion to the extra expenses generated by choosing this option?
\end{enumerate}

% These sections are not finished!!
\subsection{System prototyping and evaluation}
\label{sec:method:prototyping}
Based on the information from market research of potential solutions we matched systems with opinions and expressed need for the different parts of the organization. After collecting and creating more complete prototypes they were evaluated according to the same criteria as the individual parts (see section \ref{sec:method:research}) as well as integrational aspects. For a successful system all criteria had to be fulfilled for each part of the system necessary while providing the same or better level of service compared to what already exists. The prototypes were later presented to our supervisors at KTH as well as RWHS for response and tweaking, according to needs as well as feasibility.







% Present your results/findings.
% Make sure that your results meet the purpose and goal of the project.
% Be sure to distinguish between Results (what you found out/achieved) and Discussion.
% (Discussion is included in the next section)

% THIS IS WHAT WE SAID IN THE METHOD
% The three primary research topics/questions that we collected information about was:
% 
% How similar organizations satisfy their digital needs?
% What is the viability of creating a new custom solution within the budget requirements? 
% What are some potential alternatives for tools that are not functioning efficiently?
%
% For each of the tools researched we assessed it according to the following categories
%  How does the jurisdiction regarding secure data handling (PUL/GDPR) limits the usage?
%  Would a switch to this tools create any significant improvement?
%  Would the gain, in efficiency as well as happyness among volunteers, stand in proportion to the extra expences generated by choosing this option?

\section{Results}
In this section the results and findings from the investigation are presented. The method for conducting and evaluating our research are found above, in section \ref{sec:method}. First we briefly introduce the answers from interviews with peoples in different positions in the organization. After that we present our proposed solutions and how well they fit according to defined evaluation criteria.

% This is the table that contains all cost proposals
\begin{table*}[!t]
\centering
\caption{Sammanfattning av resultat}
\label{tab:cost-proposals}
\begin{tabularx}{\linewidth}{XXXXXXX} % llllll
\toprule \\
& Fokusgrupp 1    & Fokusgrupp 2           & Fokusgrupp 3                                                  & ----               & ---         &                                          \\
\toprule \\
Nöjda med den grafiska designen av hemsidan   		& Partially                           & No                                                               & Yes                           & Yes                                                   & Yes                                      \\
\midrule \\
Var intresserade av att klicka vidare och förstå mer     	& Partially                           & Yes                                                              & Yes                           & Yes                                                   & Yes                                      \\
\midrule \\
Tyckte det fanns för mycket funktionaliteter/verktyg     & 4 +                                 & 4 +                                                              & 2 +                           & 1                                                     & 4 +                                      \\
\midrule \\
---- & 0                                   & 0                                                                & \textgreater 50000 SEK         & \textgreater 50000 SEK                                 & 0                                        \\
\midrule \\
Monthly fee         & 0                                   & 0                                                                & 1000 SEK                       & \textless 200 SEK                                      & $\sim$ 35 SEK /user                       \\
\midrule \\
Pros                & No change required                  & Familiar to most users, all services needed included, quite simple & Efficient/fit for purpose     & Very Efficient, everything could be integrated        & Familiar to most users, simple interfaces \\
\midrule \\
Cons                & Complicated, not compliant with laws & Many different tools are needed                                  & Expensive, new system to learn & Expensive, new system to learn, requires hiring someone & Many different tools are needed \\  
\bottomrule
\end{tabularx}
\end{table*}

\subsection{Interviews}
In this section we briefly introduce the answers given by each of the interviewees. More detailed transcripts or summaries can be found in the appendices. 

\subsubsection{Fredric Landqvist}
\label{sec:method:fredrik-landqvis}

Fredric explained that one of the primary reasons for the choice of SuiteCRM was that users can work with relationships between the parties, yet also because RWHS could internally handle personal data whilst still complying with current regulations.

Regarding the problems with today's solution, Fredric addressed that the greatest disadvantage of SuiteCRM is that it is too big, in the form of presenting to much functionality to users. A lighter version of it would be better, as it would be easier for new and non technical users to use. 

\subsubsection{Henrik Resare}
% The interview with  Henrik Resare (Sales Manager at Easit)  gave a more clear view of the process Easit sets up for a new cooperation and how they can meet Refugees Welcome demands. 

Henrik explained that Easit takes care of all the technical logic that has to be developed and that it would be no problem for Easit’s solution to interact with RWHS's web page or Wordpress. Furthermore he told us that Easit works with various types of organizations and that they can always make customized workflows depending on the organizations demands. Regarding security and the PUL law, Easit stores all data in Sweden and already have functionality for the new updated GDPR law. 

\subsubsection{Malin Averstad Ryd}
Malin explained that she is now used to having many tools, but it would be better if they were fewer. Malin also reflected on that she is the only member in Stockholm using Slack, since the volunteers are only interested in matching hosts with tenants, and due to limitations such as age and lack of smart phones with volunteers. She does however inform her team about the discussions that are conducted on Slack, using emails, text messages, and when a face to face contacts occurs.

Malin also talked about the importance of Google Drive to refer to documents, as well as that the CRM that RWHS are using is not functioning properly as it should.
% like how other CRMs used by RWH. 

\subsubsection{Sara Hadfy Högström}
Sara Hadfy Högström explained in her interview that she is relatively satisfied with the workflow in her region but that she probably uses too many services herself (around 12 different).  The full transcript is available in Appendix \ref{app:sara-hadfy} but a short summary is presented below.

She felt that the workload was decreasing and that it would be hard to justify an investment in new and expensive systems. As the future financing of the organization is uncertain an extra monthly fee would also be hard to motivate. Her main complaints of the current situation were regarding SuiteCRM being hard to learn, having technical issues and not always working as intended.

The team that she was working with were at the time using WhatsApp for daily communication and Trello for handling the landlords. Data was periodically transferred to Trello from SuiteCRM because Trello was more efficient, gave a better overview and was easier for collaboration. She could agree on switching to G Suite instead of Trello if needed however expressed that it would be unreasonable to work solely on SuiteCRM.

% This is what we promised to provide in form of background research
% How similar organizations satisfy their digital needs?
% What is the viability of creating a new custom solution within the budget requirements? 
% What are some potential alternatives for tools that are not functioning efficiently?

% For each of the tools researched we assessed it according to the following categories
% How does the jurisdiction regarding secure data handling (PUL/GDPR) limits the usage?
% Would a switch to this tools create any significant improvement?
% Would the gain, in efficiency as well as happyness among volunteers, stand in proportion to the extra expences generated by choosing this option?

\subsection{Solutions on the market}
In this section, we present the findings about each of the different alternative solutions that we found for RWHS. A summary of the primary aspects of interest are presented in table \ref{tab:cost-proposals}.

\subsubsection{Current Solution}
\label{sec:result:current-solution}
% As the the research was gathered an understanding was made that the current solution was not legal. The current solution did not follow the PUL law since it was gathering some information about people in an structured way in Google Drive. A change is therefore needed, especially with the GDPR law in mind. 
% The research and interviews showed that the current CRM is very “messy” and is not contributing in the sense of how technology and ICT solution should contribute to efficiency in daily work. 
The current solution, which consists of a combination of different tools, has problems with both legal issues as well as usability. This means that it is not an optimal choice and some changes are necessary immediately to avoid fines. It should be a priority to conduct the following three changes:
\begin{enumerate}
\item Stop using Trello as it does not comply with laws regarding personal data (PUL/GDPR) (see section \ref{sec:background:trello}).
\item Make sure that documents on Google Drive are only shared with users that require them, to make sure that it complies to PUL and GDPR. This includes both read and write access.
\item Ensure that RWHS comply with GDPR in all ways before 25th of May 2018.
\end{enumerate}
To further improve the current system we found most users to agree that the following changes would improve the experience with the system.
\begin{enumerate}
\item Clean interface and remove unnecessary functionality
\item Reduce number of data items visible during regular use. In the default mode, only contact information related to relevant registrations should be visible.
\item Make sure that the system runs quicker and has fewer technical problems/bugs.
\item Add easier ways of following the development of the situations with landlords in each region. 
\end{enumerate}

% After interviews and gathered information we saw as a result that the current solution needs to be improved: 
% We have concluded this in mainly three different areas that is presented as a result if current solution should stand. These points should be improved.
% \begin{enumerate}
% \item Clean current platform (CRM)
% \item Invest in education. Better documentation
% \item Remove Trello since it is not legally aligned with GDPR
% \item Make sure to follow the GDPR law.
% \end{enumerate}

% Furthermore we have concluded that since the current solution is not legal it shall be replaced and these are the other options that we would like to present as solutions. 

\subsubsection{Easit and other CRMs}
A new CRM with a less cluttered interface and functionality that is customized to RWHS demands is optimal. We found multiple different providers offering this service however all which of them were expensive (above 50 000 SEK). We decided too look into one of the options called Easit. This solution could still be an option as investing in the system has a lot of advantages. One advantage is that Easit takes care of all the functionality, support and education. They have also made it clear that the integration from RWHS's web page to Easit's systems will not be a problem. These were the two greatest concerns and challenges Fredric Landqvist addressed, provided that RWHS would change system (more information about his interview can be found in section \ref{sec:method:fredrik-landqvis}). Easit's system is available in two versions, as a service with a monthly fee or as product that a organization could buy. In the second case only a smaller fee for support and maintenance would be payed each month.

% Easit offers two types of solutions, one solution is renting their services and systems. By renting, RWHS have to pay a monthly fee. The other solution is to buy Easit's services and systems, doing this a higher one time cost have to be payed but then a much lower monthly fee is payed for support and maintenance. 

% This belongs in the discussion
% This could be a problem because as said RWH is a non-profit organization.
% However this solution comes with a price, integrating RWHS:s data and making an external system compatible with RWHS:s web page could be expensive. 
% If RWHS could apply for some kind of economic contribution, buying the solution is a better option as we think the monthly payment for this is not as significantly as it would have been if the organization decides to rent.      

\subsubsection{G Suite}
\label{result:google-suite}
% On the 25th May of 2018 the European Union will implement the GDPR. This law make it possible for companies outside of EU to contribute as long as they are following the GDPR law. The law will strengthen the rights that individuals have and unify all EU residents. Regardless of where the data is processed the GDPR law seeks to unify the data protection for all EU residents. Therefore it is possible for companies outside of EU to handle data about individuals, and have their servers in “third-countries” as long as they are following GDPR.  


The collection of G Suite services could provide RWHS with all necessary tools for conducting daily tasks, including collecting new contact information where the CRM previously was used. This could be done using the tool \textit{Google Forms} which automatically collects user data into spreadsheets which would be available to the volunteers through \textit{Google Drive}. In addition to providing the necessary tools, all of them also complies with the new GDPR standard as presented in section \ref{sec:background:g-suite}. A switch to this tool chain would simplify the daily workflow and reduce the total number of tools used in different parts of the country. At the same time the tools would be familiar to most users and not provide unnecessarily complex interfaces. A negative aspect is how data is stored. As all registrations would be kept in spreadsheets, it could be \changeRemove{difficile}\changeAdd{difficult} to search through the data and store it longer periods.

% The only thing needed is an account providing the domain name that the organization want to use with Google services. Once this is done the services are available, which includes Gmail, Calendar, Drive, and other core G Suite services. 

% For RWHS Google offers a free Google Suite for nonprofit organization. Refugees Welcome would eligible, but needs as a first step do an application: 

% Our recommendation is that in the website you simply do a google form, where you register and then the registration will automatically go into a google drive. This step means you will skip CRM and make a simplified work flow.

\subsubsection{Custom Solution}
A perfect solution does not exist on the market. A perfect solution would therefore be a custom solution specifically produced for RWHS. A custom system would be customized for RWHS's specific requirements, their workflow and the data handled. It would be an expensive solution to create, therefore one option is to start from an already existing system (suiteCRM) and configure the system to better fit  RWHS's needs, which is discussed in section \ref{sec:result:current-solution}.

% The workflow for RWHS is very specialized and does not really fit into a category where tools already exists on the market for their specific purpose. This means that the best would be to create a custom tool for handling data in the way that the organization is currently doing. Such a system could include all information and support for the workflow. This type of system would be expensive to create and one option is to start from an already existing system (suiteCRM) and configure it to fit better with the needs which is discussed in section \ref{sec:result:current-solution}.

\subsubsection{Microsoft Office 365 Enterprise (for non-profits)}
One alternative to G Suite is the office package offered by Microsoft. It offers approximately the same functionality, whilst offering more recognizability with users. What is lacking is the option to integrate a form creation tool, which exists in G Suite with Google Forms. Microsoft, as Google, also offer their system with a reduced cost for non-profit organizations. One feature from Microsoft that is absent in Google Suite is a web app called Microsoft Tasks which could be a Trello replacement that still follows GDPR as presented in section \ref{sec:background:microsoft-office}. 





\section{Discussion and Conclusion}
\label{sec:discussion}
In this chapter we present the conclusion from the project, and discuss the method and result presented in previous chapters. We also reflect on  the projects limitations, presented solutions and the implications this project can have on sustainability factors. 

\subsection{Limitations}
During this project there were a set of limitations to consider. The project contained two different types of limitations, school limitations and organizational limitations. Because this mainly was a school project we had to adapt our method and result to fit the timescale and goals within the course. As RWHS is a non-profit organization we had strict economical limitations as well.

Due to these restrictions there is a chance that our result is not optimal for other organizations and due to the time limit there could be other solutions on the market that has not been researched and analyzed. Most non free solutions were simply not considered.

\subsection{Pros and cons of solutions}
Using only productivity as a factor, using Easit or another costly CRM would be the best solution. Easit would be an easily integrated solution which would require little modification of the current work flow. Unfortunately cost is a big restriction, which eliminates Easit as an alternative.

Keeping the current solution, with small alterations to ensure compliance with GDPR as well as reorganizing the current CRM, would be cheap both considering money and effort. It would however not completely solve the problem regarding the usability of SuiteCRM. It would only slightly reduce the usability problems, however not require a great adjustment of work flow, which is an advantage.
%A big advantage would be as previously mentioned, low cost in both time and effort. Additionally it would not require great adjustment of work flow.

To use only G Suite would be a slightly bigger change, however all of the problems regarding CRM usability would be mostly solved. It would require minimal effort to change the current registration link from the CRM form to a Google Form and have the data collected into spreadsheets automatically. Some regions that today use Trello for landlord management would have to switch to using a Google Sheet, however the regions that already use G Suite for landlord management would instead automatize their work flow, by eliminating the CRM as a middle step, and then manually moving the data to Google Drive.

A switch to Microsoft's office suite does not feel motivated as it would increase the costs of the organization, yet not add much new functionality. At the same time there is a new system to implement and learn which would not be optimal.

\subsection{Sustainability}
Achieving sustainable growth requires changes in many traditional industrial processes. The technology should make everyday work easier in companies and organizations \cite{Vergragt}. In RWHS an automation of the tools would contribute to a more efficient workflow and more motivation among the volunteers. This would hopefully result in more work towards the goals of the organization, which addresses very relevant sustainability factors in our society today. By contributing to the work environment for RWHS we hope to indirectly affect the same sustainability goals that they are directly affecting.

As we only have an indirect effect on this aspect, there are no guarantees that a greater number of landlords and refugees would be matched based only on providing more effective digital tools for RWHS. On the other hand, having good tools is\changeRemove{ really} a prerequisite for performing a good job, which \changeRemove{means that an upgraded system would at least maximize}\changeAdd{would suggest that an upgraded system could maximize} their abilities to do what they are doing as good as possible.

This chapter discusses some of the most important sustainability factors that RWHS are contributing to. The focus is social sustainability, sharing economy and UN's sustainability goals which are discussed in sections \ref{sec:discussion:social-sustainability}, \ref{sec:discussion:sharing-economy} and \ref{sec:discussion-sdg} respectively.

% This is already in the report in the problem section
% \textbf{Are we successfully investigating and solving our problems?}
% \textbf {Over separation of internal and external information sharing, caused by a fragmented system of services. Easy communication, both internal and external, is needed to organize workers and inform them about procedures and regulations. The current solution stores and communicates information across a multitude of platforms which are hard to synchronize. 
% Usage problems for new as well as more experienced volunteers. The volunteers are required to learn a multitude of services, just for get started. This time is lost to technical details instead of starting on the important work. It disconnects the volunteers from working at the major issue. Beyond being an extra time waste, it creates a hindrance for volunteers and decreases morale within the organization as the current procedure is way too inefficient.
% Secure handling of personal data. RWHS handles personal and sensitive data from refugees, as well as landlords. Substantial parts of this information is lawfully required to be securely kept, which has to be considered when proposing a digital solution.}

\subsubsection{Social sustainability}
\label{sec:discussion:social-sustainability}
Since RWHS is a non-profit organization and the volunteers are working without pay, the aim with the technology should be to optimize the volunteers work and make it as interesting as possible. Internally the tools should contribute to efficient work so that the volunteers can focus on work tasks that are fun and where they feel that they contribute in an effective way. This would improve their work quality and in a long term perspective give a work flow that could include more volunteers and make them perform better.

The definition of social sustainability has been widely discussed but the one we introduced in section \ref{sec:background:sustainability} (\textit{\nameref{sec:background:sustainability}}) is aligned with RWHS's work. RWHS are finding housing for refugees who do not know anyone in the Swedish society yet, which contributes to the refugees situation by giving them an improved social and culture life that only a personal connection in a new place will give you. At the same time \changeAdd{some of} the refugees are \changeRemove{mostly}\changeAdd{, according to Kajsa Sörman,} using preexisting resources (spare rooms), which improves the efficiency at which heated space is used.

The social sustainability contributions that RWHS are doing could also be classified as bridge sustainability. RWHS with their platform encourages potential landlords to give up otherwise unused space, and by doing that helps the potential landlords to contribute to the bio-physical environment as a consequence. This is a non-transformative approach as it doesn't require changes in either life style or believes.
 
\subsubsection{Sharing Economy}
\label{sec:discussion:sharing-economy}
By creating new opportunities to create closer relationships between tenants and landlords, RWHS is also creating a rich climate for a sharing economy where parts of apartments could be shared if single rooms are rented by refugees. From a sustainable point of view it is positive for people and communities to share some of their resources. One example is household appliances such as ovens, laundry machines etc. that are only used a fraction of the time. Instead of producing machines for every individual it is more efficient to share them among many.

%% We already said that these to sections discuss how we indirectly affect
%% sustainability factors. It is redundant information to add this again.
% If RWHS implements one of the provided solutions this project will have an indirect effect on the \textit{Sharing Economy} perspective. By making the required tools and the work flow more simple, the volunteers can work more effective and as a result match more landlords with refugees.

%% Yes, as previously mentioned... We don't need to say this again.
% As mentioned the solutions this project offers  only have an indirect effect on this aspect and there is no guarantee that more landlords and refugees will be matched based only on providing more effective digital tools for RWHS.

\subsubsection{Sustainable Development Goals}
\label{sec:discussion-sdg}
As previously mentioned in the background, the most relevant of UN's SDGs in this specific case are:
\begin{itemize}
\item Goal 10 - Reduced Inequalities
\item Goal 11 - Sustainable cities and communities 
\item Goal 12 - Responsible consumption and production
\item Goal 16 - Peace, justice and strong institutions
\end{itemize}
We introduced these goals since we believe these are some of the goals that RWHS work is aligned with. We will present all of the goals and why RWHS work is widely important. 

\textit{Goal 10 - Reduced Inequalities} 
RWHS main goal is to connect newcomers with hosts that offers housing. This matching supports the integration of newcomers with the Swedish society. Such integration could also develop the social aspects of the refugees towards the society, by understanding the culture and traditions. This hopefully makes them feel more equal and connected with other members of the Swedish society.

Beside the social inclusion, RWHS helps with economic inclusion as well. By quickly providing newcomers accommodation, this helps them settle and allows them to focus on having better chances of employment and higher income, which contribute to having equal economy, comparing to other members in the society.

\textit{Goal 11 - Sustainable cities and communities} 
RWHS is in alignment with goal 11 since they enable more efficient use of available housing. For instance, if all young adults had an apartment it would not be sustainable. As section \ref{sec:discussion:sharing-economy} brings forward, sharing resources is necessary to have sustainable cities and communities. 
\cite{UnitedNationsb}

\textit{Goal 12 - Responsible consumption and production}, by matching together people and having them share housing and resources the organization consequently works toward this goal. This SDG and the impact RWHS has on it can be related to the aspects that section \ref{sec:discussion:sharing-economy} (\textit{\nameref{sec:discussion:sharing-economy}}) brings forward.

\textit{Goal 16 - Peace, justice and strong institutions}. As mentioned in section \ref{sec:background:sdg} (\textit{\nameref{sec:background:sdg}}) this goal includes working against physical and sexual abuse of individuals. In 2016 \textit{Kalla fakta}, one of Swedens biggest investigative journalism programs revealed a series of serious abuse of refugees in a refugee accommodation in Dalarna \cite{TV4KallaTv4.se}. The same year Expressen revealed similar events taking place on another accommodation \cite{RoosAsylrapporten:Asylboendet}. News of this variety occur on a periodically basis, both about refugees and in other types of accommodations where vulnerable individuals are located. The Swedish Migration Agency has been criticized for not controlling and taken care of these kinds of problems. To prevent problems of this kind, RWHS interviews the landlords to detect any possible warning signs. RWHS also value the relationships between the refugees and the landlords very highly. By providing RWHS with a more effective tool they can focus on these parts more and hopefully prevent abuse in the homes refugees end up living in.  
 
\subsubsection{Goal conflicts}
As the result has shown, RWHS follows several of the SDG, however there are goal conflict that could occur such as a conflict between Goal 10 and Goal 12. We have by now understood RWHS's work and what they do. If we only respond to their sharing resources as goal 12 - Responsible consumption and production their work for helping refugees could be seen as only helping one minority. This is a widely discussed topic for those who feel that 'favoring' a minority in Sweden is not right, where there is a housing crisis. There are students in Sweden that can not find apartments and RWHS is giving a service to find housing for a specific minority. Concluding this could unfortunately be a goal conflict with goal 10.

\subsection{Final recommendation}
Considering both requirements and limitations, our final recommendations are that RWHS should develop a standard for how the workflow looks in all teams, and make sure that this standard is compliant with GDPR. This should be done without changing the way of working in the teams too much, as it could create a hindrance for them. A standard for workflow in all teams would both be easier to manage on a national level and would also make it possible to verify that it complies with relevant laws. As most teams seem to exclude the current CRM from their workflow, the easiest option at the moment would be to temporarily move completely to G Suite by changing the registration procedure as described in section \ref{result:google-suite}. A more long term sustainable way of handling and storing information would be to invest time (and money) into customizing and cleaning the current CRM system. There is already knowledge of this system within the organization which makes it one of the best choices for a starting point for a custom system. Other custom (or already existing CRM systems) are usually way too expensive, whilst still not perfectly satisfying the organizations needs.

% \begin{enumerate}
% \item \textbf{Google Suite}
% \newline Google, you can link it in wordpress.
% https://www.google.com/cloud/security/gdpr/ 
% Google takes responsibility for law suits? 
% \item \textbf{Clean current platform}
% \newline A cleaning up would make it better, make sure it follows the GDRP law, and invest in education- better documentation! For example make a monthly cleaning day- "tech- day", so it is not in the daily task but something everyone are contributing in monthly.
% \end{enumerate}


\printbibliography

\onecolumn
\newpage
\twocolumn
\appendices
\section{Personal Information}
\label{app:personal-information}
This is the definition of personal information by Datainspektionen:
"All slags information som direkt eller indirekt kan hänföras till en fysisk person som är i livet räknas enligt personuppgiftslagen som personuppgifter. Även bilder (foton) och ljudupptagningar på individer som behandlas i dator kan vara personuppgifter även om inga namn nämns. Krypterade uppgifter och olika slags elektroniska identiteter, som exempelvis IP-nummer, räknas som personuppgifter om de kan kopplas till fysiska personer." \cite{Datainspektionen2015}

\section{Easit}
\label{app:easit}
Easit is a Swedish software company that develops systems and processes that creates solutions for keeping companies and organizations operation in order. Easit's case management system enables you to streamline your business by keeping track of all issues and deviations in your organization. Easit was founded in 1999 and have been using web technology, the product becomes accessible to anyone who has a browser and customers can be offered a flexible business model for both licensing and operation. Easit uses its own personnel for development, deployment, training and support - giving us great opportunities to adapt to customer requirements. All development and support takes place in Sweden. \cite{Easit}

\section{Henric Resare}
\label{app:henric-resare}
Questions and answers from Henric Resare

\subsection{How is the setup process done from a technical point of view?}
Easit takes care of all the technical logic that’s need to be done. No problem for us to interact with RWHS web page or Wordpress. 

\subsection{RWHS does not sell anything and don’t have “regular” customers, do you have a solution that is more for following processes?} 
We can make customized workflow depending on the organizations demands.

\subsection{How much would your solution cost?}
We could try to make our service cheaper because it is a non profit organization but it will cost about 5000 SEK/month if they want to rent. Buying the system solution will be over 50 000 SEK (one time cost) and a monthly cost for support. 

\subsection{RWHS have to follow the PUL-law, how do you work according to that?}
We store all data in Sweden and already have functionality for the new updated GDPR-law that will come in May 2018. 

\section{Fredric Landqvist}
\label{app:fredric-landqvist}
Questions and answers from Fredric Landqvist

\subsection{Why did you choose to work with SuiteCRM?}
The first groups in Germany used a case management system instead called OSTicket, but they sorted the process in Excel. OSTicket and Excel did not work smoothly with lots of data. In Sweden we got a sponsored web server and there was a CRM package included. In CRM you can work with relationships between the parties. But also because we could internally handle personal data (PUL).

\subsection{Have you (or someone else) had any training or workshop for the volunteers using the CRM?}
No. Difficult with larger training, as some volunteers work for an hour just sometimes and it is very common people quit and new volunteers join. Some volunteers may only want to match one specific refugee, not optimal for them to learn a complex system.

\subsection{How does SuiteCRM work with the PUL-law?}
As long as individual users have their own accounts and you have control over who has access, we are within the framework of PUL. Server should preferably be in Sweden but the EU is okay. The server today is located in Holland.

\subsection{Your personal pros and cons about SuiteCRM?}
Disadvantage: Too big, a light version would been good, we are aiming to start scaling down functionality.
Advantage: You can follow the flow.

\subsection{What you think would be the hardest challenge if RWHS would decide to change system and how the store data?}
Moving over the data and learning users the new system will be the biggest challenges, even to get the integration with Wordpress to work.

\subsection{How much are you willing to pay for another solution?}
Almost nothing, free is good. Would like to have an open source.

\section{Interview with Malin Averstad Ryd}
\label{app:malin-averstad}
Questions and answers from Malin Averstad Ryd.

\subsection{What services are you currently using in your team?}

\begin{enumerate}
\item Slack (Only Malin)
\item GSuites (Google Drive, Gmail)
\item Text Messages
\item Personal Meetings
\item Trello
\item CRM
\end{enumerate}

\subsection{For each service: what do you use it for? What is the work flow?}

\subsubsection{Slack}
 
Malin is the local coordinator for Stockholm Regional team. Slack is mainly used for communications with other regional teams and for organizational purposes.

 \subsubsection{Google Drive}
Google drive is used to refer to the important documentations. Such a documentations are volunteer agreements, project planes and flayers and more. 

\subsubsection{Gmail}
Gmail is used to send the volunteer agreements and guidelines to the volunteers. Malin Also informs her team with updates happening on Slack by e-mails. 

\subsubsection{Text Messages}
Text messages are used on daily basis for communications between the team members. Ordinary text messages are used because most of the team members in Stockholm, doesn't have a smart phone, as well as due to age limitation. 

\subsubsection{Personal Meetings}
Most of the team members likes to conduct personal meeting, to agree on the work flow and assign tasks.

 \subsubsection{Trello}
Trello is used to assign tasks, and track the progress of a matching tenants with hosts.
 
 \subsubsection{CRM}
CRM is only used to collect the contact the information for the registered hosts (personal information, rent price, and description for the property such as: size, shared or private, etc.). 
 

 
\subsection{What are your most important tools for the work? Prioritize them!}

\begin{enumerate}
\item GSuites (Google Drive, Gmail)
\item Slack
\item Text Messages
\end{enumerate}
 
\subsection{What are your main annoyances in the current setup.}
CRM is not functioning properly, and that CRM had a history function, where a progress of contacting landlord can be tracked.
 
Having a lot of services for the same purpose, where Malin got used to having different tools. But many volunteers would like to have a one platform to be used. 
 
\subsection{What would you remove if you could?}
Trello, since it is only used when having an extensive workload, and for the moment she is not active as she was before.
 
\subsection{Do you have any suggestions for tools to try out instead?}
Having a face to face contact, because most of the people fear handling with platforms and tools, which are new to them and might not be user friendly.

Refugees Welcome International teams are using an integration between Google accounts with CRM. Such an integration allows all the communication that are carried with the hosts to have a record on the CRM, as well as discussions, emails, and contact information. 



\section{Interview with Sara Hadfy Högström}
\label{app:sara-hadfy}
Sara Hadfy Högström is the only employee of RWHS. Everybody else are volunteers. Sara works on a national level, however she mostly oversee a lot of the work in Malmö, where RWHS is most active.

\subsection{What services are you currently using in your team?}
Some services are only used by Sara personally in her role as a project manager.
\begin{enumerate}
\item Slack
\item WhatsApp
\item Google Drive
\item Google Hangouts
\item Gmail
\item Google Calendar
\item SuiteCRM
\item Trello
\item MailChimp
\item Telephone (only Sara)
\item Facebook (only Sara)
\item Skype (only Sara)
\end{enumerate}

\subsection{For each service: what do you use it for? What is the workflow?}
In general all volunteers get access to all channels. The more active a volunteer is, the more channels a volunteer can choose to use.

\subsubsection{Slack}
Slack is mostly used for contacting other active volunteers. Slack is used for organisational work, however not day-to-day work. As a project leader Sara uses Slack a lot. The volunteers in Malmö barely use Slack.
\subsubsection{WhatsApp}
WhatsApp is used for day-to-day coordination of the volunteer team in Malmö. The team is a small team of about 7 people. It works really good for its purpose. Another advantage is that many volunteers already use WhatsApp. Sara gets a more personal connection with the volunteer group, which she feels make it more enjoyable to be a volunteer, and less like work. Sara has annoyances with uploading files on WhatsApp, however she seldom does it. It happens about twice a year.

\subsubsection{Google Drive}
Google drive is used for all kinds of documents. That can be meeting protocols, pamphlets, flyers, receipts and more. Sara feels that it is a bit difficult to know where to put documents, however it recently got a lot better. The organisation recently executed a major reorganisation of their Google Drive. They recently started limiting users to only have access to a specific folder that is only purposed for that volunteers specific region. Sara claims that it simplifies usage of the Google Drive for the volunteers. The Google Drive is not used very much by the volunteer team in Malmö.

\subsubsection{Google Hangouts}
Hangouts is used for meetings. It is considered very easy to use.

\subsubsection{Gmail}
All volunteers get a personal gmail with their google suites account. Most volunteers do not use their gmail for mail at all. Sara does in her role as project manager.

\subsubsection{Google Calendar}
There is no coordinated shared calendar for the volunteers to use. Sara uses her to keep track of everything RWHS related. Most volunteers probably do not use this. Sara use Google Calendar to invite volunteers to meetings.

\subsubsection{SuiteCRM}
SuiteCRM is used only by Sara in Malmö, and she checks it once a week and moves all relevant data to Trello. SuiteCRM feels very non user friendly, and partly broken to Sara. Some functionality is only available to very few people, and if Sara wants to use that functionality, Sara has to contact one of these administrators.

\subsubsection{Trello}
Trello is used as a replacement for SuiteCRM. It is allegedly easier for volunteers to use and adds functionality as easier state management, greater overall visibility and an easy to use discussion thread for every landlord.

\subsubsection{MailChimp}
Only some people use MailChimp, whereas Sara is not one of them. It’s used for only external communication, and often that is newsletters. The recipients are mostly old volunteers or other really interested people who are very interested of RWHS.

\subsubsection{Telephone (only Sara)}
Sara uses her personal telephone both to call volunteers and external parties like municipalities and government agencies. Sara also uses her telephone to send sms messages. When a previously interested landlord does not answer on telephone or email, Sara sends a last text message, informing the landlord that they are always welcome back if they change their mind.

\subsubsection{Facebook (only Sara)}
Sara is administrator for the RWHS facebook group. Sometimes people contact RWHS through the facebook group. In those cases Sara is one of them who answers.

\subsubsection{Skype (only Sara)}
There exists a skype number, which is a normal telephone number that redirects to Saras phone. That is the official number on the RWHS website. Very few people call this number, however Sara feels that it is nice to have a telephone number that people can call.

\subsection{What are your most important tools for the work? Prioritize them!}
\begin{enumerate}
\item GSuites (Google Drive, Google Hangouts, Gmail, Google Calendar)
\item Slack
\item WhatsApp
\item Trello / CRM. Does not use one without the other
\item Facebookkonto
\item Skype
\item Mailchimp
\end{enumerate}

\subsection{What are your main annoyances in the current setup?}
It would be better if there were fewer services. In the beginning of every day Sara open all services and log in.

The workflow is pretty good as it is, however it takes time to learn it. It is probably most annoying for new volunteers, however it is a great way to choose a level of engagement in the organisation. Volunteers can use more services if they feel that they want to be more engaged.

\subsection{What would you remove if you could?}
It would be nice if SuiteCRM worked like Trello. It would be nice if SuiteCRM worked better.

\subsection{Do you have any suggestions for tools to try out instead?}
No.

\subsection{Other thoughts}
When a member joins the organisation they sign a contract. This contract is mostly for ensuring that they use their digital access responsibly. One detail on this contract that could change is that it says that if a volunteer is inactive for two weeks, their membership will be revoked. A problem is that this is not applied in reality. Members can sometimes be inactive up to half a year before getting their membership revoked.

The work burden is pretty low, which in turn makes it pretty unjustified to dedicate resources and time on improving the digital workflow.

It is not reasonable that the organisation has a large monthly cost for any solution, because the long term economic situation is very unclear. A one time payment is much more justified.

Using SuiteCRM without using Trello or Google Drive as it is used today is impossible.

\section{Legal Disclaimer}
This disclaimer governs the use of this report. By using this report, you accept this disclaimer in full.

\subsection{Not legal advice}
This report contains information about GDPR and PUL. The information is not legal advice and should not be treated as such.

You must not rely on the information in the report as an alternative to legal advice from an appropriately qualified professional. If you have any specific questions about any legal matter you should consult an appropriately qualified professional.

You should never delay seeking legal advice, disregard legal advice, or commence or discontinue any legal action because of information in the report.

\subsection{No representations or warranties}
To the maximum extent permitted by applicable law we (the authors of the report) exclude all representations, warranties, undertakings and guarantees relating to the report.

Without prejudice to the generality of the foregoing paragraph, we do not represent, warrant, undertake or guarantee:
\begin{itemize}
\item that the information in the report is correct, accurate, complete or non-misleading;
\item that the use of guidance in the report will lead to any particular outcome or result.
\end{itemize}

\subsection{Liability}
We will not be liable to you in respect of any losses arising out of any event or events beyond our reasonable control.

We will not be liable to you in respect of any business losses, including without limitation loss of or damage to profits, income, revenue, use, production, anticipated savings, business, contracts, commercial opportunities or goodwill.

We will not be liable to you in respect of any loss or corruption of any data, database or software.

We will not be liable to you in respect of any special, indirect or consequential loss or damage.

\subsection{Severability}
If a section of this disclaimer is determined by any court or other competent authority to be unlawful and/or unenforceable, the other sections of this disclaimer continue in effect.  

If any unlawful and/or unenforceable section would be lawful or enforceable if part of it were deleted, that part will be deemed to be deleted, and the rest of the section will continue in effect.

%This report is not legal advice. We (the authors of the report) take no legal responsibility for any of the information being legally accurate. 


% \input{sections/6-legal-disclaimer}

\end{document}
