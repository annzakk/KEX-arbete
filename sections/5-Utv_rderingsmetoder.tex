\section{Utförande - (working title)}
I det här kapitlet beskrivs det praktiska tillvägagångssättet mellan metod och resultat. Först presenteras utformningen av enkät, och dess undersökning följt av genomförandes av denna.
Fokusgrupperna beskrivs analogt. 
\subsection{Utformning av enkätundersökning}
När enkätundersökningen utformades hade vi Laugwitz, Held, och Schrepp's arbete \enquote{Construction and Evaluation of a User Experience
Questionnaire} som riktlinje och mall\cite{Laugwitz2008ConstructionQuestionnaire}. Deras mall användes eftersom att ramverket föddes ur en djup forskning inom området UX, och var avsedd för applikationer så som våran, det vill säga extraheringen av information från en befintlig produkt/prototyp. Användarupplevelsen delas upp i delar vilket Laugwitz et al. kallar för kategorisering (se mer om detta under sektionen Kategorisering \ref{kategorisering})\cite{Laugwitz2008ConstructionQuestionnaire}.\\

Boynton och Greenhalgh menar på att en nyckelpunkt när man utformar ett frågeformulär är att använda tidigare validerade formulär\cite{Boynton2004SelectingQuestionnaire}. Således var utgångspunkt mallen \enquote{Construction and Evaluation of a User Experience
Questionnaire} och sedan formades en enkät med frågor för att få svar på det som besvarar studiens forskningsfråga och är i studiens omfång.  

\subsubsection{Kategorisering}
Eftersom att frågor ofta är vinklade kan man ställa samma fråga fast på olika sätt för att få ett mer exakt svar. Det var på detta sätt som frågorna arbetades fram enligt Laugwitz et al. \cite{Laugwitz2008ConstructionQuestionnaire}. Kategoriseringen som skall täcka hela användarupplevelsen enligt Laugwitz et. al \cite{Laugwitz2008ConstructionQuestionnaire} gör att man kan ta fram en skala för empirisk data. Skalan är från 1 till 6 där respondenten kan välja ett alternativ på varje fråga. Varje kategori hade liknande frågor som ställdes på olika sätt för att säkerställa respondentens svar och sen kunna ta fram en total bedömning på den användarupplevelsen av AL1 (se hela enkäten i Bilaga B). Läs mer om detta under sektionen Kategorisering \ref{kategorisering}.
\newline

Frågorna som ställdes i enkätundersökningen var: 

\begin{enumerate}
\item \textbf{Attraktivitet:}
\newline
Jag upplever att prototypens totala intryck är:
\begin{itemize}
\item Ful/Snygg
\item Irriterande/Angenäm
\item Otrevlig/Tilltalande
\newline
\end{itemize}

\item \textbf{Effektivitet:}
\newline
Mitt totala intryck av prototypen är att den går/är:
\begin{itemize}
\item Långsam att använda/Snabb att använda
\item Ineffektiv/Effektiv
\item Opraktisk/Praktisk
\item Rörig/Välorganiserad
\newline
\end{itemize}


\item \textbf{Pålitlighet:}
\newline
Jag upplever prototypen som:
\begin{itemize}
\item Oförutsägbar/Pålitlig
\item Komplicerad/Enkel
\item Förvirrande/Tydliga instruktioner
\newline
\end{itemize}


\item \textbf{Tydlighet:}
\newline
Jag upplever prototypen som
\begin{itemize}
\item Svår att förstå/Lätt att förstå
\item Hämmande/Underlättande
\item Oviktig/Viktig
\newline
\end{itemize}

\item \textbf{Effektivitet:}
\newline
Jag upplever prototypen som
\begin{itemize}
\item Omodern/Modern
\item Oinspirerande/Inspirerande
\item Jobbig/Bekväm
\newline
\end{itemize}

\end{enumerate}

I bilaga B kan det fullständiga frågeformuläret hittas. 

\subsection{Genomförande av enkätundersökning}
Enkäten skapades i Google Forms \cite{GoogleForms}, detta för att det är ett enkelt system för utformning av en enkät och dessutom väldigt enkelt att sprida, med hjälp av en länk. När utkastet till enkäten var färdigställd så testades den på ett antal utomstående personer där de fick gå igenom prototypen och svara på enkäten. Vid testningen av enkäten klarades frågetecken ut och revidering efter kritik verkställdes. Denna feedback var viktig för att försäkras om att frågorna var ställda på ett sådant sätt att betraktaren skulle förstå varje fråga och för att datan som samlades in inte skiljde sig mellan respondenterna. 
\newline

Enkäten skickades ut till relevanta personer, via Linkedin och e-mail. I beskrivningen presenterades vilka vi var och syftet med enkäten. Enkätundersökningen genomfördes digitalt tillsammans med prototypen och tillhörande frågor om den. 
\newline

\subsubsection{Målgrupp (Urval)}
Målgruppen för respondenter har varit framtida slutanvändare till applikationen. När urvalet av personer som skulle få möjligheten att svara på enkätundersökningen gjordes resonerades det fram till att personer inom en organisation som vill lära sig mer om EQ var de som var lämpliga att svara på enkäten. 
\newline

Urvalet har utgått från Amazing Leaders' kontakter och sedan varit ett snöbollsurval\cite{RankinKvalitativaMetoder}. Snöbollsurval är ett icke-slumpmässigt urval av personer som man med hjälp av tidigare valda personer letar sig fram till andra personer som man kan ha med i urvalet. \cite{VejdeSnobollsurval} Detta var lämpligt då Amazing Leaders har haft ett brett nätverk där de i sin tur vetat andra som passar målgruppen. Enkäten skickades ut till sammanlagt 60 personer.
\label{sec:perspektiv}


\subsection{Utformning av fokusgrupp}
%När man utformar en fokusgrupp ska man vara medveten om vissa aspekter som: 
%\begin{itemize}
%\item Problemformuleringen 
%\item Målgruppen (urval) 
%\item Genomförande 
%\item Analysarbetet
%\end{itemize}
%\subsubsection{Problemformulering}
När man utformar en fokusgrupp vill man i sin bredaste grad få svar på hur användaren upplever prototypen, vilket har varit utgångspunkten av vår studie. Vi vill få fram utmaningar, helhetsupplevelsen och spekulera kring framtida förbättringar. För denna typ av studie resonerade det sig i att utformningen bör ha en samtalsledare och därför lämpar sig semi-strukturerade och ostrukturerade intervjuer bäst \cite{Denscombe2010TheProjects}. En semi-strukturerad intervju innebär att frågor skrivs som ett bedömningsunderlag där man under intervjuns gång kan fråga relevanta följ frågor, vilket lämpade sig för det skapade känslan av ett samtal vilket ökar  diskussionen \cite{RankinKvalitativaMetoder}. Att jobba med en ostrukturerad intervjuteknik är att man utfår från ett blankt papper där respondenten får styra samtalet\cite{PaUtvarderingsmetodikExperience}. En växelverkan mellan dessa metodiker ansågs passa bäst då den ostrukturerade intervju-tekniken gör att forskaren kan vara moderator med syfte att ingripa så lite som möjligt under intervjun för att tillåta gruppen diskutera fritt. Genom att utföra enkätundersökningarna först får vi förutsättningar att upptäcka vilka områden inom UX som var \enquote{gap-filling} och i fokusgrupperna undersöka utövandet av UX med detta i åtanke, se mer om vad \enquote{gap-filling} innebär under Metod~\ref{sec:background} \cite{Alvesson1985GENERATINGPROBLEMATIZATION}. 
\newline

\subsubsection{Kategorisering}
Precis som enkätundersökningarna följer fokusgrupperna kategoriseringen av Laugwitz et al \cite{Laugwitz2008ConstructionQuestionnaire}. Studiens målsättning har varit att ha ett kvalitativt tillvägagångsätt för att komplettera enkätundersökningens resultat. Läs mer om detta under sektionen Kategorisering \ref{kategorisering}. 
\newline

Frågor som ställdes på tillfällena var: 
\newline
\begin{enumerate}
\item \textbf{Attraktivitet}
\begin{itemize}  
\item Vad tycker du om det allmänna utseendet av prototypen, är den inbjudande att använda rent utseendemässigt?
\item Varför?
\item Varför inte?
\item Vad var bra och varför? 
\item Vad var dåligt och varför?
\item Hur viktigt är det för din användarupplevelse? Rösta från 1/10 till 10/10, skriv ditt svar på denna lapp
\end{itemize}

\item \textbf{Tydlighet}
\begin{itemize}  
\item Hur tydlig är prototypen, är det lätt att förstå vad som förväntas av dig som användare? Är det lätt att förstå hur man tar sig till nästkommande steg?
\item Varför?
\item Varför inte?
\item Vad var bra och varför? 
\item Vad var dåligt och varför?
\item Hur viktigt är det för din användarupplevelse? Rösta från 1/10 till 10/10, skriv ditt svar på denna lapp
\end{itemize}

\item \textbf{Pålitlighet}
\begin{itemize}  
\item Hur pålitlig(känns den seriös/oseriös) känns prototypen? Känns den trovärdig? 
\item Varför?
\item Varför inte?
\item Vad var bra och varför? 
\item Vad var dåligt och varför?
\item Hur viktigt är det för din användarupplevelse? Rösta från 1/10 till 10/10, skriv ditt svar på denna lapp
\end{itemize}

\item \textbf{Stimuli}
\begin{itemize}  
\item Hur modern är den, är den “i tiden”? Känns den rolig/stimulerande att använda? 
\item Varför?
\item Varför inte?
\item Vad var bra och varför? 
\item Vad var dåligt och varför?
\item Hur viktigt är det för din användarupplevelse? Rösta från 1/10 till 10/10, skriv ditt svar på denna lapp
\end{itemize}

\item \textbf{Effektivitet}
\begin{itemize}  
\item Hur effektiv och snabb kändes den? Var den praktisk och organiserad? 
\item Varför?
\item Varför inte?
\item Vad var bra och varför? 
\item Vad var dåligt och varför?
\item Hur viktigt är det för din användarupplevelse? Rösta från 1/10 till 10/10, skriv ditt svar på denna lapp
\end{itemize}
\end{enumerate}

\subsection{Genomförande av fokusgrupp}
I samarbete med Amazing Leaders kom vi fram till att ha tre olika tillfällen i deras lokaler för att genomföra fokusgrupperna. För att generera en positiv ton och öka intresset hos deltagarna och för att säkerställa att deltagarna dök upp gjorde vi ett frukostevenemang, lunchevenemang och ett kvällsevenemang där vi bjöd på mat och dryck. Som tidigare diskuterat resonerade vi att mindre fokusgrupper skulle ge deltagarna mer utrymme för diskussion och ge möjlighet för alla att komma till tals. Grupperna formades efter våra kriterier, se sektionen Målgrupp nedan (5.4.1). Vi ansåg att en bra diskussion skulle kunna fortlöpa om deltagarna inte kände varandra eller jobbade på samma arbetsplats. Anledningen till bakom varför vi valde att forma grupperna på detta har att göra med att vi ville bibehålla alla deltagarnas relationsmässiga integritet. Vi ville inte att tidigare relationer och redan etablerade gruppdynamik skulle reflekteras i de svar som samlades in, och att deltagarna istället skulle få utrymma att tala fritt utan inverkan av extern påverkan. 
\newline


\subsubsection{Målgrupp (Urval)}
\label{malgrupp}
För att i en undersökning kunna dra slutsatser där data är validerad behöver man ta hänsyn till området, målgrupp\cite{Gualtieri2009BestDesign}. Vid möten med Amazing Leaders förstod vi att de hade ett stort urval av kontakter som skulle vara slutanvändaren. Då Amazing Leaders har en stor kundkrets behövdes det göras ett urval inför fokusgrupperna. Det urval som gjordes inför fokusgrupperna baserades på kriterierna: 
\begin{itemize}
\item Ålder
\item Bakgrund 
\item Yrkesverksamma år
\item Position i företaget 
\end{itemize}

Kartläggningen och ansatsen gjorde att vi fick ett diversifierat urval av erfarenhet, kön, anställning, bakgrund, utbildning, yrkesverksamma år och position i företaget. 

\subsection{Analys} 
För att kunna utvärdera den insamlade data, och avgöra vilka aspekter av användarupplevelsen som är viktigast, enligt respondenterna, presenteras insamlad data i stapeldiagram. Eftersom att datan som är insamlad är en indikation på hur viktig en kategori är för en bildar stapeldiagrammen bra översikt för hur fördelningen ser ut i gruppen som deltog i enkätundersökningen. Vidare kunde slutsatser om hur viktiga de olika kategorierna var för deltagarna i undersökningen dras. 
\\

För fokusgrupperna har vi använt oss av ljudinspelningar och anteckningar. Fokusgrupperna varade i en timme vardera där vi transkribera materialet på detaljnivå. Efter det började vi kategorisera och hitta gemensamma nyckelord och påståenden som var snarlika varandra. Vi anser att detta var ett viktigt stadium för att kunna strukturera upp och få fram data för vår forskningsfråga. Viktiga och meningsfulla diskussions ämnen redovisas som citat, där citaten tilldelas till relevant kategoriseringsämne. Se hela diskussionerna i Bilaga B.
\newline


