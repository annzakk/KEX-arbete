\onecolumn
\newpage
\twocolumn
\appendices
\section{Personal Information}
\label{app:personal-information}
This is the definition of personal information by Datainspektionen:
"All slags information som direkt eller indirekt kan hänföras till en fysisk person som är i livet räknas enligt personuppgiftslagen som personuppgifter. Även bilder (foton) och ljudupptagningar på individer som behandlas i dator kan vara personuppgifter även om inga namn nämns. Krypterade uppgifter och olika slags elektroniska identiteter, som exempelvis IP-nummer, räknas som personuppgifter om de kan kopplas till fysiska personer." \cite{Datainspektionen2015}

\section{Easit}
\label{app:easit}
Easit is a Swedish software company that develops systems and processes that creates solutions for keeping companies and organizations operation in order. Easit's case management system enables you to streamline your business by keeping track of all issues and deviations in your organization. Easit was founded in 1999 and have been using web technology, the product becomes accessible to anyone who has a browser and customers can be offered a flexible business model for both licensing and operation. Easit uses its own personnel for development, deployment, training and support - giving us great opportunities to adapt to customer requirements. All development and support takes place in Sweden. \cite{Easit}

\section{Henric Resare}
\label{app:henric-resare}
Questions and answers from Henric Resare

\subsection{How is the setup process done from a technical point of view?}
Easit takes care of all the technical logic that’s need to be done. No problem for us to interact with RWHS web page or Wordpress. 

\subsection{RWHS does not sell anything and don’t have “regular” customers, do you have a solution that is more for following processes?} 
We can make customized workflow depending on the organizations demands.

\subsection{How much would your solution cost?}
We could try to make our service cheaper because it is a non profit organization but it will cost about 5000 SEK/month if they want to rent. Buying the system solution will be over 50 000 SEK (one time cost) and a monthly cost for support. 

\subsection{RWHS have to follow the PUL-law, how do you work according to that?}
We store all data in Sweden and already have functionality for the new updated GDPR-law that will come in May 2018. 

\section{Fredric Landqvist}
\label{app:fredric-landqvist}
Questions and answers from Fredric Landqvist

\subsection{Why did you choose to work with SuiteCRM?}
The first groups in Germany used a case management system instead called OSTicket, but they sorted the process in Excel. OSTicket and Excel did not work smoothly with lots of data. In Sweden we got a sponsored web server and there was a CRM package included. In CRM you can work with relationships between the parties. But also because we could internally handle personal data (PUL).

\subsection{Have you (or someone else) had any training or workshop for the volunteers using the CRM?}
No. Difficult with larger training, as some volunteers work for an hour just sometimes and it is very common people quit and new volunteers join. Some volunteers may only want to match one specific refugee, not optimal for them to learn a complex system.

\subsection{How does SuiteCRM work with the PUL-law?}
As long as individual users have their own accounts and you have control over who has access, we are within the framework of PUL. Server should preferably be in Sweden but the EU is okay. The server today is located in Holland.

\subsection{Your personal pros and cons about SuiteCRM?}
Disadvantage: Too big, a light version would been good, we are aiming to start scaling down functionality.
Advantage: You can follow the flow.

\subsection{What you think would be the hardest challenge if RWHS would decide to change system and how the store data?}
Moving over the data and learning users the new system will be the biggest challenges, even to get the integration with Wordpress to work.

\subsection{How much are you willing to pay for another solution?}
Almost nothing, free is good. Would like to have an open source.

\section{Interview with Malin Averstad Ryd}
\label{app:malin-averstad}
Questions and answers from Malin Averstad Ryd.

\subsection{What services are you currently using in your team?}

\begin{enumerate}
\item Slack (Only Malin)
\item GSuites (Google Drive, Gmail)
\item Text Messages
\item Personal Meetings
\item Trello
\item CRM
\end{enumerate}

\subsection{For each service: what do you use it for? What is the work flow?}

\subsubsection{Slack}
 
Malin is the local coordinator for Stockholm Regional team. Slack is mainly used for communications with other regional teams and for organizational purposes.

 \subsubsection{Google Drive}
Google drive is used to refer to the important documentations. Such a documentations are volunteer agreements, project planes and flayers and more. 

\subsubsection{Gmail}
Gmail is used to send the volunteer agreements and guidelines to the volunteers. Malin Also informs her team with updates happening on Slack by e-mails. 

\subsubsection{Text Messages}
Text messages are used on daily basis for communications between the team members. Ordinary text messages are used because most of the team members in Stockholm, doesn't have a smart phone, as well as due to age limitation. 

\subsubsection{Personal Meetings}
Most of the team members likes to conduct personal meeting, to agree on the work flow and assign tasks.

 \subsubsection{Trello}
Trello is used to assign tasks, and track the progress of a matching tenants with hosts.
 
 \subsubsection{CRM}
CRM is only used to collect the contact the information for the registered hosts (personal information, rent price, and description for the property such as: size, shared or private, etc.). 
 

 
\subsection{What are your most important tools for the work? Prioritize them!}

\begin{enumerate}
\item GSuites (Google Drive, Gmail)
\item Slack
\item Text Messages
\end{enumerate}
 
\subsection{What are your main annoyances in the current setup.}
CRM is not functioning properly, and that CRM had a history function, where a progress of contacting landlord can be tracked.
 
Having a lot of services for the same purpose, where Malin got used to having different tools. But many volunteers would like to have a one platform to be used. 
 
\subsection{What would you remove if you could?}
Trello, since it is only used when having an extensive workload, and for the moment she is not active as she was before.
 
\subsection{Do you have any suggestions for tools to try out instead?}
Having a face to face contact, because most of the people fear handling with platforms and tools, which are new to them and might not be user friendly.

Refugees Welcome International teams are using an integration between Google accounts with CRM. Such an integration allows all the communication that are carried with the hosts to have a record on the CRM, as well as discussions, emails, and contact information. 



\section{Interview with Sara Hadfy Högström}
\label{app:sara-hadfy}
Sara Hadfy Högström is the only employee of RWHS. Everybody else are volunteers. Sara works on a national level, however she mostly oversee a lot of the work in Malmö, where RWHS is most active.

\subsection{What services are you currently using in your team?}
Some services are only used by Sara personally in her role as a project manager.
\begin{enumerate}
\item Slack
\item WhatsApp
\item Google Drive
\item Google Hangouts
\item Gmail
\item Google Calendar
\item SuiteCRM
\item Trello
\item MailChimp
\item Telephone (only Sara)
\item Facebook (only Sara)
\item Skype (only Sara)
\end{enumerate}

\subsection{For each service: what do you use it for? What is the workflow?}
In general all volunteers get access to all channels. The more active a volunteer is, the more channels a volunteer can choose to use.

\subsubsection{Slack}
Slack is mostly used for contacting other active volunteers. Slack is used for organisational work, however not day-to-day work. As a project leader Sara uses Slack a lot. The volunteers in Malmö barely use Slack.
\subsubsection{WhatsApp}
WhatsApp is used for day-to-day coordination of the volunteer team in Malmö. The team is a small team of about 7 people. It works really good for its purpose. Another advantage is that many volunteers already use WhatsApp. Sara gets a more personal connection with the volunteer group, which she feels make it more enjoyable to be a volunteer, and less like work. Sara has annoyances with uploading files on WhatsApp, however she seldom does it. It happens about twice a year.

\subsubsection{Google Drive}
Google drive is used for all kinds of documents. That can be meeting protocols, pamphlets, flyers, receipts and more. Sara feels that it is a bit difficult to know where to put documents, however it recently got a lot better. The organisation recently executed a major reorganisation of their Google Drive. They recently started limiting users to only have access to a specific folder that is only purposed for that volunteers specific region. Sara claims that it simplifies usage of the Google Drive for the volunteers. The Google Drive is not used very much by the volunteer team in Malmö.

\subsubsection{Google Hangouts}
Hangouts is used for meetings. It is considered very easy to use.

\subsubsection{Gmail}
All volunteers get a personal gmail with their google suites account. Most volunteers do not use their gmail for mail at all. Sara does in her role as project manager.

\subsubsection{Google Calendar}
There is no coordinated shared calendar for the volunteers to use. Sara uses her to keep track of everything RWHS related. Most volunteers probably do not use this. Sara use Google Calendar to invite volunteers to meetings.

\subsubsection{SuiteCRM}
SuiteCRM is used only by Sara in Malmö, and she checks it once a week and moves all relevant data to Trello. SuiteCRM feels very non user friendly, and partly broken to Sara. Some functionality is only available to very few people, and if Sara wants to use that functionality, Sara has to contact one of these administrators.

\subsubsection{Trello}
Trello is used as a replacement for SuiteCRM. It is allegedly easier for volunteers to use and adds functionality as easier state management, greater overall visibility and an easy to use discussion thread for every landlord.

\subsubsection{MailChimp}
Only some people use MailChimp, whereas Sara is not one of them. It’s used for only external communication, and often that is newsletters. The recipients are mostly old volunteers or other really interested people who are very interested of RWHS.

\subsubsection{Telephone (only Sara)}
Sara uses her personal telephone both to call volunteers and external parties like municipalities and government agencies. Sara also uses her telephone to send sms messages. When a previously interested landlord does not answer on telephone or email, Sara sends a last text message, informing the landlord that they are always welcome back if they change their mind.

\subsubsection{Facebook (only Sara)}
Sara is administrator for the RWHS facebook group. Sometimes people contact RWHS through the facebook group. In those cases Sara is one of them who answers.

\subsubsection{Skype (only Sara)}
There exists a skype number, which is a normal telephone number that redirects to Saras phone. That is the official number on the RWHS website. Very few people call this number, however Sara feels that it is nice to have a telephone number that people can call.

\subsection{What are your most important tools for the work? Prioritize them!}
\begin{enumerate}
\item GSuites (Google Drive, Google Hangouts, Gmail, Google Calendar)
\item Slack
\item WhatsApp
\item Trello / CRM. Does not use one without the other
\item Facebookkonto
\item Skype
\item Mailchimp
\end{enumerate}

\subsection{What are your main annoyances in the current setup?}
It would be better if there were fewer services. In the beginning of every day Sara open all services and log in.

The workflow is pretty good as it is, however it takes time to learn it. It is probably most annoying for new volunteers, however it is a great way to choose a level of engagement in the organisation. Volunteers can use more services if they feel that they want to be more engaged.

\subsection{What would you remove if you could?}
It would be nice if SuiteCRM worked like Trello. It would be nice if SuiteCRM worked better.

\subsection{Do you have any suggestions for tools to try out instead?}
No.

\subsection{Other thoughts}
When a member joins the organisation they sign a contract. This contract is mostly for ensuring that they use their digital access responsibly. One detail on this contract that could change is that it says that if a volunteer is inactive for two weeks, their membership will be revoked. A problem is that this is not applied in reality. Members can sometimes be inactive up to half a year before getting their membership revoked.

The work burden is pretty low, which in turn makes it pretty unjustified to dedicate resources and time on improving the digital workflow.

It is not reasonable that the organisation has a large monthly cost for any solution, because the long term economic situation is very unclear. A one time payment is much more justified.

Using SuiteCRM without using Trello or Google Drive as it is used today is impossible.

\section{Legal Disclaimer}
This disclaimer governs the use of this report. By using this report, you accept this disclaimer in full.

\subsection{Not legal advice}
This report contains information about GDPR and PUL. The information is not legal advice and should not be treated as such.

You must not rely on the information in the report as an alternative to legal advice from an appropriately qualified professional. If you have any specific questions about any legal matter you should consult an appropriately qualified professional.

You should never delay seeking legal advice, disregard legal advice, or commence or discontinue any legal action because of information in the report.

\subsection{No representations or warranties}
To the maximum extent permitted by applicable law we (the authors of the report) exclude all representations, warranties, undertakings and guarantees relating to the report.

Without prejudice to the generality of the foregoing paragraph, we do not represent, warrant, undertake or guarantee:
\begin{itemize}
\item that the information in the report is correct, accurate, complete or non-misleading;
\item that the use of guidance in the report will lead to any particular outcome or result.
\end{itemize}

\subsection{Liability}
We will not be liable to you in respect of any losses arising out of any event or events beyond our reasonable control.

We will not be liable to you in respect of any business losses, including without limitation loss of or damage to profits, income, revenue, use, production, anticipated savings, business, contracts, commercial opportunities or goodwill.

We will not be liable to you in respect of any loss or corruption of any data, database or software.

We will not be liable to you in respect of any special, indirect or consequential loss or damage.

\subsection{Severability}
If a section of this disclaimer is determined by any court or other competent authority to be unlawful and/or unenforceable, the other sections of this disclaimer continue in effect.  

If any unlawful and/or unenforceable section would be lawful or enforceable if part of it were deleted, that part will be deemed to be deleted, and the rest of the section will continue in effect.

%This report is not legal advice. We (the authors of the report) take no legal responsibility for any of the information being legally accurate. 
