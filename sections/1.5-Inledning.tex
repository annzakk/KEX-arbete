\section{Inledning}
Internets utveckling är en av de starkaste krafterna som har förändrat vår värld de senaste två decennierna\cite{Davidsson2017Svenskarna2017}. Dagligen möts vi av olika webbapplikationer som vi interagerar med, både i arbetslivet och på fritiden\cite{Six2011TheUXmatters}. Således är det viktigt att användaren kan interagera med applikationen på ett så enkelt och bra sätt som möjligt \cite{Six2011TheUXmatters}. För att detta ska ske är en av grundprinciperna att sätta användaren i fokus\cite{Six2011TheUXmatters}. Genom att ta reda på vad användaren tycker och vad som kan förbättra användarupplevelsen skapas ett större värde för applikationen\cite{Rodden2010MeasuringApplications}. Oavsett om applikationen ska användas av privatpersoner eller internt på ett företag ger information om användarupplevelsen en viktig och värdefull åsikt som kan bidra till förbättringar \cite{Rodden2010MeasuringApplications}.
\newline

\parindent
\parskip

\subsection{Bakgrund}
Datorsystem är en typ av interaktivt system och karakteriseras av ett signifikant antal interaktioner mellan människa och dator \cite{InteractiveDictionary}. Interaktiva system är något som behöver vara användbart, lättförståeligt och som får användaren till fortsatt användning\cite{May2012ApplyingApp}. Interaktiva system är således något som måste utgå från människocentrerad design\cite{Human-CenteredStudio}. Människocentrerad design utgår från fysiska och psykiska behov hos mänskliga användare\cite{Millot2014Human-CenteredDesign}. Människocentrerad design blandas ofta ihop med termen användarcentrerad design\cite{Abras2004User-CenteredDesign}. Användarcentrerad design fokuserar på en djupare analys av målgruppen, specifika egenskaper och specifika egenskaper hos målgruppen, denna typ av design är något som fokuserats på i denna rapport.
\newline

Människa- datorinteraktion (MDI) är ett forskningsområde som studerar interaktionen mellan människa och dator\cite{Blanton2009Human-ComputerInteraction}. Forskningsområdet etablerades på tidigt 60-tal men det var inte förrän på 80-talet som termen Människa- datorinteraktion fick sin spridning\cite{Myers1998ATechnology}. Idag ses MDI och dess begrepp som en nödvändighet för att utforma interaktiva system\cite{TheCambridge}. 
\newline

Ett nyckelkoncept inom MDI är något som kallas \textit{användbarhet}\cite{Myers1998ATechnology}. Begreppet är en mätning på hur väl användaren kan interagera med eller använda ett system \cite{Blanton2009Human-ComputerInteraction}. För att uppnå en god användbarhet krävs det att man arbetar med människocentrerad design där man arbetar med principer som effektivitet, minnesvärdighet och tillfredsställelse\cite{Blanton2009Human-ComputerInteraction}. 
\newline

Ett annat centralt begrepp som etablerat sig inom MDI på senare tid är User Experience (UX). User Experience direkt översatt till svenska är \textit{användarupplevelse}. UX är alla aspekter av slutanvändarens interaktion med företaget, dess tjänster och dess produkter enligt Norman\cite{NormanTheUX}, forskare inom människa- datorinteraktion. UX kan appliceras inom alla branscher, produkter eller tjänster, på alla lösningar där användaren är i fokus\cite{Axbom2013JagAxbom}. UX som forskningsområde syftar till användarens totala upplevelse av en produkt eller tjänst\cite{20Mastery}. 
\newline

Då teknik har blivit så pass etablerat i vårt samhälle där interaktiva system har blivit en stor del av människans vardag är det viktigt att interaktionen med systemen sker på bästa möjliga sätt \cite{Six2011TheUXmatters}. Den största utmaningen vad det gäller denna utveckling är hur man får användaren till fortsatt användning\cite{SafferCreatingDevices}. Det är många faktorer som spelar in vid utveckling av ett lyckat interaktivt system. Bland dessa faktorer spelar användbarhet och UX en stor roll\cite{SafferCreatingDevices}. På endast ett fåtal år har samsynen på UX blivit mer uppmärksammad där man har fått större förståelse för att upplevelsen för användaren kan gå förlorad under projektets olika stadier fram till att produkten är klar\cite{BrannmarkGustavJansaterHandledareInstitutionenInformatik}. Genom att utveckla systemen utefter ett människocentrerat perspektiv är chansen större att systemet tillfredsställer användarens behov och bidrar till fortsatt användning\cite{May2012ApplyingApp}. 

%Denna rapport kommer att utgå från ett teoretiskt ramverk för UX som metodik för att beskriva olika dimensioner av användarupplevelsen. Detta ramverk är framtaget av Marc Hassenzahl \cite{Hassenzahl2001TheAppealingness} där målet är att förstå vilken kategori slutanvändaren anser vara både viktigast för användarupplevelsen men också som en kartläggning för utvärdering av användarupplevelsen. Kategoriseringen som tagits fram och som även används i denna rapport är: \enquote{Attraktivitet, Tydlighet, Stimuli, Pålitlighet och Effektivitet} \cite{Laugwitz2008ConstructionQuestionnaire}.

\subsection{Uppdragsarbetet}
Kandidatarbetet utförs tillsammans med \textit{VNTRS} \cite{VNTRSVNTRSAB} åt \textit{Amazing Leaders}\cite{AmazingOrganisationsutveckling}. VNTRS är ett konsultbolag som startades år 2016 av fyra management konsulter. VNTRS har i nuläget (2018) 21 anställda som arbetar med att hjälpa företag att bygga innovativa digitala produkter. VNTRS arbetar bland annat tillsammans med Amazing Leaders, som är en av deras uppdragsgivare.
\newline
\parindent
\parskip

Amazing Leaders är en organisation som grundades av tidigare industriledare som nu coachar och utbildar sina kunder. Utbildningens syfte är att öka ledareffektivitet. Ledareffektiviteten utvecklas med hjälp av metoder inom forskningsområdet EQ (Emotional Quotient / emotionell kvot) och coachningen går ut på att höja denna kvot. Amazing Leaders avser i framtiden att erbjuda utbildningsprogram där programmen delvis utförs digitalt via en app, eller där digitala appar finns tillgängliga som hjälpmedel under programmets gång. VNTRS och Amazing Leaders samarbetar idag för att uppnå denna vision. 
\newline
\parindent
\parskip

Som del av utvecklingsarbetet för att uppnå deras mål har VNTRS utvecklat en prototyp. Denna prototyp var avsedd för att testa olika koncept kring hur effektiva olika utbildningsmetoder var för slutanvändaren. Mycket av behovet som finns hos VNTRS och Amazing Leaders, i relation till detta arbete, är baserat på att från Amazing Leaders prototyp kunna dra slutsatser som är nyttiga i det slutgiltiga arbetet - vilket är att skapa en användarvänlig och användarcentrerad applikation.

\subsection{Problemformulering} 
Problemet som ger upphov till denna studie är att VNTRS, och i utsträckning Amazing Leaders inte vet exakt i vilken riktning man vill ta utvecklingen av deras applikationsprototyp för att den ur ett UX perspektiv skall vara så användarcentrerad som möjligt. Tillsammans vill man därför utföra en studie som kan  ge riktlinjer kring hur man kan fortsätta i det befintliga utvecklingsarbetet.
\newline 

Kondenserat blir forskningsfrågan som följande: \\

Vad tycker slutanvändaren om prototypen och hur förbättrar man användarupplevelsen för Amazing Leaders framtida applikation? 



% Studien har som målsättning att göra användartestning på AL1 för att först vad användaren tycker om prototypen. Problemet är att skapa en applikation som når upp till användarens förväntningar och är användarcentrerad. Det kräver att man i ett tidigt skede förstår målgruppen som inte varit bestämd. Därav är ett problem som studien tacklar är vilken målgruppen det är. 

%För att göra testning för att förstå användarupplevelsen behövs en prototyp som testas på användaren där det finns  en väl fungerande funktionaliteten{\cite{Gualtieri2009BestDesign}. Problemet har alltså varit att utveckla prototypen så den inte är så pass begränsad att det påverkar det estetiska tilltalet.

% Studier visar på att olika applikationer har mer eller mindre krav beroende på syftet med användandet av applikationen. Dessa krav berör också upplevelsen som användaren har. Således behöver man behärska både syftet med applikationen och vad slutanvändaren anser vara viktigast för användarupplevelsen \cite{5Foundation}.Frågeställningen rapporten vill besvara är vilken del som är viktigast för användarupplevelsen samt vad man tycker är bra eller dåligt med prototypen AL1. 
% \newline

% En jämförelse på prototypen skall göras med testning på målgruppen. Frågeställningarna som ska besvaras är:

% \subsection{Frågeställningar}
% \begin{enumerate}
% \item Vad är bristfälligt med prototypen (AL1)?
% \item Vad som anses vara bra med AL1 enligt slutanvändaren?
% \item Vad är viktigast för användarupplevelsen för Amazing Leaders framtida applikation enligt kategoriseringen framtagen av Marc Hazzendahl?
% \end{enumerate}

\subsection{Syfte}
Syftet med detta arbeta är att testa  användarupplevelsen för Amazing Leaders' prototyp. Studien ska också avgöra vad som är viktigast för användarupplevelsen av prototypen enligt slutanvändaren.

\subsection{Mål}
Målet är att samla in data som kan användas för att identifiera vad som är viktigt för Amazing Leaders' slutanvändare. Med denna data vill man kunna förstå vad slutanvändaren anser vara mest värdeskapande och bristfälligt.
Målsättningen är vidare att VNRTRS ska förstå vad som ska prioriteras för att gå vidare i utvecklingen av deras digitala applikation.

% Syftet med denna studie är att med hjälp av användartestning på Amazing Leaders prototyp, AL1, utvärdera vad som är mest värdeskapande och bristfälligt med prototypen. Målet är förstå vad personerna som ska använda applikationen tycker är bristfälligt, nöjdhetsfaktorer samt vad som är viktigast för användarupplevelsen. Resultatet av studien kommer att lyfta fram betraktarens användarupplevelse i form av en procentuell beräkning om vad hen anser vara viktigast för användarupplevelsen. 

\subsection{Samhällsnytta, etik och hållbarhet}
Vanligtvis när en applikation utvärderas från ett människa-dator perspektiv, gör man förändringar i gränssnittet för att göra den mer intuitivt och enklare att använda. Genom att vara med och förbättra gränssnittet kommer Amazing Leaders' slutanvändare ha en mer positiv användarupplevelse. Hur vida detta bidrar till någon bredare samhällsnytta är svårt att diskutera inom ramarna av ett teknologiskt arbete. 
\newline

Det råder inga tvivel om att det som Amazing Leaders' försöker åstadkomma genom självledarskap och ökandet av EQ har en positiv inverkan, om inte minst i slutmottagaren av applikationen. I ett större perspektiv, är de slutsatser vi drar i fältet MDI, oavsett hur små eller stora, bidragande till att utveckla och öka förståelsen inom fältet, för andra eller oss själva i framtiden. Nya tillvägagångssätt ökar värdeskapande för både andra företag som till forskningen. Denna applikation kommer användas i utbildningssyfte och ger där av en viss kravställning, där man arbetar etiskt så att användaren får en objektiv synvinkel och rätt information av EQ.
\\

Hållbar utveckling innehåller tre delar, den ekonomiska, den sociala och den ekologiska utvecklingen \cite{KTHHallbarKTH}. Användandet av applikation kan komma minska det fysiska mötet vilken påverkar den fysiska mötesformen som är miljövänligare då det minskar transport och utsläpp, detta har positiv inverkan framförallt ur miljöaspekter. 
\newline

Direkta ekonomiska aspekter innefattar en direkt förbättrad verksamhet för både VNTRS och Amazing Leaders. De kunder som de tar del av Amazing Leaders tjänster har som mål att på något sätt förbättra sin verksamhet. Vare sig detta är givet som en positiv inverkan i ett perspektiv i samhällsnytta, etik och hållbarhet är svårt att avgöra. Många företag idag har som del av affärsidén som mål att bedriva en grön och samhällsetisk verksamhet, bl.a genom att tillägna resurser till CSR-relaterat arbete. Det man kan hoppas är att Amazing Leaders i framtiden har ett selektivt tankesätt när det gäller deras kunder och samarbetspartners, och väljer att besläktat sig till organisationer som tar i hänsyn hållbar utveckling. 
\newline

Någon uppenbar social inverkan finns inte heller till synes. Det man kan argumentera för är att en god användarupplevelse förbättrar en människas upplevelse vid användning. Syftet med deras applikation är till stor del att få människor att må bättre och att ledare även ska förstå andra människor bättre (se forskningsområdet EQ). Detta resonemang är fortfarande på en lokal nivå. Ett bredare narrativ skulle kunna vara att lärdomarna som kan fås av att utveckla en väldigt bra app för Amazing Leaders leder till att andra företag även kan ta efter och fortsätta trenden att coacha och utbilda digitalt. Vare sig detta är positivt eller negativt är ett forskningsområde i sig själv. 

% I ett arbete med mobila applikationer bör det tas hänsyn till de sociala hållbarheten vilket är att skapa en applikation som har en god användarupplevelse för att förbättra människors upplevelse vid användning. Syftet med applikationen är att få människor att må bättre, och att ledare ska förstå andra människor bättre. Genom att få vara med i kedjan av att förbättra och skapa har vi en indirekt påverkan som skapar välmående hos andra vilket gör att vårt arbete har en rekyl effekt (rebound effekt) som är i målsättning med hållbar utveckling \cite{EnergieffektiviseringarForetagsvarlden}. Under studiens gång har dessa aspekter tagits till hänsyn där vår förhoppning är att arbeta i parallella vägar till hållbar utveckling. 


\subsection{Introduktion till metod}
I mån av att uppnå syftet av rapporten antas ett vetenskapsteoretiskt arbetssätt som kallas triangulering. Trianguleringen som metod innefattar en litteraturstudie, enkätundersökningar och fokusgrupper. Men den kollektivqa kunskap som kan anförskaffas genom trianguleringen förs abduktiva resonemang för att dra slutsatser för att uppnå rapportens mål. 
\newline

En prototyp är framtagen för detta ändamål som vi löpande under rapportens gång har kallat Amazing Leaders' Prototyp 1 (\enquote{AL1}). Prototypen är central i denna rapport och har utvärderats med hjälp av bland annat fokusgrupper för att ta reda på användarens åsikter om  AL1. I AL1 får användaren ett smakprov på en övning inom självledarskap. Användarens upplevelse har bedömts efter Hassenzahls ramverk för UX \cite{Hassenzahl2001TheAppealingness} och kategorisering av Laugwitz et al. \cite{Laugwitz2008ConstructionQuestionnaire} som bedömer användarupplevelse efter fem attribut, nämligen: Attraktivitet, Tydlighet, Stimuli, Pålitlighet och Effektivitet. 
\newline


AL1 är ett sätt för Amazing Leaders att utveckla en applikation som tillåter digilog (digital digilog), som är en användarcentrerad interaktion mellan coach och aspirant. Således är AL1 ett redskap för metodvalen som gjorts. Det som vill uppnås med rapporten är att ta lärdom från fokusgrupper samt med hjälp av kvantitativ datainsamling (enkätundersökningar) förstå vad som är 

\begin{enumerate}
\item Bristfälligt med prototypen, och
\item Vad som anses vara bra med prototypen enligt slut-användaren.
\item Vad är viktigast för användarupplevelsen för Amazing Leaders applikation?
\end{enumerate}
Vidare kommer slutsatser dras kring hur Amazing Leaders prototyp upplevs av användaren (målgruppen) i nuläget samt vilket attribut som tycks vara den mest betydelsefulla för användaren i en applikation kontra hur detta attribut anammat sig på AL1.  


\subsection{Målgrupp}
\label{sec:problem-definition}
Vetrov visar på att konstitutionen av framgångsrika produkter har skett där man validerar och löser rätt problem åt rätt målgrupp och på rätt marknad. \cite{Vetrov2013AppliedUXmatters} 
Just eftersom att, som beskrivet under problemformuleringen, det inte finns någon slutgiltig fastställning av hur den framtida applikation som Amazing Leaders försöker skapa skall se ut är det också svårt att helt definiera den slutgiltiga målgruppen. 
\newline

Man kan avgränsa studien och arbeta enligt ett antal grundprinciper i relation till målgruppen för att få en så bra uppfattning om den som möjligt. Det vi vet är att Amazing Leaders i dagsläget har ett stort fokus på ledarskap och att deras kundurvalsgrupp främst riktar sig mot ledare inom diverse företagssektorer. Det vi också vet är att applikationen kommer att ha ett fokus på EQ och coachning/utbildning relaterat till detta fält, kanske då främst riktat mot självledarskap och allmän företagsledning. 
\newline

Vi väljer vidare att skilja mellan målgruppen av applikationen och målgruppen som vi behöver skapa för att uppnå dom mål vi har satt för studien. Som tidigare nämnt under avsnittet (REF: mMÅL) har en av målsättningarna varit att förstå vad som är viktigt, i termer av användarvänlighet för Amazing Leaders slutanvändare utifrån den befintliga prototyp som är utvecklad. För att göra detta behövdes två målgrupper för både enkätundersökningar och fokusgrupper.
\newline

För enkätundersökningen begränsades målgruppen till personer inom befintliga organisationer och företag som vill lära sig mer, och är intresserade av greppet EQ. Dessa personer väljs också för att de i framtiden kan komma att vara användare av Amazing Leaders' framtida applikation. Eftersom att Amazing Leaders har en stor kundkrets behövdes en avgränsning av deras kundkrets och ett urval gjordes. Urvalet av för fokusgrupperna baseras på: 
\begin{itemize}
\item Ålder
\item Bakgrund 
\item Yrkesverksamma år
\item Position i företaget 
\end{itemize}


%Den ekonomiska aspekten av resultatet kommer inte tas till hänsyn, inte heller  huruvida resultatet ska implementeras. Avgränsning sker också inom hur den framtida applikationen ska marknadsföras för Amazing Leaders. För att kartlägga arbetssättet fokuseras det på om användaren har haft en god upplevelse eller inte. 


%Rapportens omfattning gör att en avgränsning har gjorts där vi inte jämför de olika utvärderingsmetoderna utan använder både kvantitativ och kvalitativ metod för att få ett mångfaldigt resultat om målgruppen kan komma att ändras. -Patrik fattade inte denna meningen. 

\subsection{Avgränsningar}

Ett av målen har också varit  att granska vad som är mest värdeskapande och bristfälligt med prototypen så har begränsningar gjorts i form av inte implementera en ny prototyp utifrån resultaten från fokusgrupperna samt enkätundersökningen. Denna begränsning gjordes för att tidsramen för detta kandidatarbete inte skulle räcka till för att göra en ny prototyp med dess förbättringar utifrån resultaten från metoderna. Inga lösningar eller krav på vad som ska implementeras i den framtida mobila applikationen kommer heller inte att presenteras i denna rapport, detta också på grund av tidsaspekten. Det som fokuserades på i denna rapport var således endast en utvärdering av AL1.


\subsection{Disposition}
I kapitel 2 presenteras teorin som studien är baserad på. Där presenteras Människocentrerad design, Kategorisering och Prototyper. Detta avsnitt presenterar väsentlig teori och hur vi har tolkat denna teori. 
\newline

I kapitel 3 presenteras uppdragsgivare, ramverk och resurser som är en bakgrund för att som läsare förstå vad projektet är baserat på. Detta kapitel ger nödvändig information för att förstår väsentliga begrepp, som prototypen AL1, vilka uppdragsgivarna Amazing Leaders är samt hur det iterativa arbetet har utförts. 
\newline

I kapitel 4 presenteras metoden som använt i denna rapport. Kapitlet inkluderar en generell förklaring av metodval samt för och nackdelar, tillsammans med hur arbetet utfördes. 
\newline

I Kapitel 5 presenteras utförandet av arbetet, framförallt införskaffandet av data från enkätundersökningen och fokusgrupperna, och hur man har använt denna data för att få fram resultat.
\newline

I kapitel 6 redovisas resultaten från både från  enkätundersökningen och fokusgrupperna, och vilka slutsatser som har dragits från resultaten. 

I kapitel 7 förs en diskussion om arbetet och dess förutsättningar samt ett avslutande avsnitt på fortsatt arbete.