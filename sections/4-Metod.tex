\section{Metod}
I den här delen av uppsatsen presenteras metodvalen som är relevanta för studien och anledning till varför dem valdes. För att få en bred uppfattning om användarens uppfattning av AL1 genomfördes denna studie enligt triangulering: \textit{litteraturstudie, fokusgrupper} och \textit{enkätundersökningar}. 

\subsection{Tillvägagångssätt}
\label{sec:background}
Studien hade som syfte att utforska den totala användarupplevelsen av AL1-prototypen. Det var viktigt att metodvalet var tydligt och konkret, där uppsättningen av krav för studien bemöttes och där aspekter från litteratur och forskningsområdet togs i åtanke. 
\newline

Studien använde metoden triangulering för att öka validiteten och har bestått av \textit{litteraturstudie, fokusgrupper och enkätundersökningar} \cite{RankinKvalitativaMetoder}. En litteraturstudie gjordes för att få en teoretisk bakgrund som krävdes för projektet samt för att kunna välja relevant utvärderingsmetoder. Det som var mest relevant för studien efter att ha analyserat teorier vid liknande studier var Hassenzahls ramverk \cite{HassenzahlUserQuality} samt Lauqwitz et al. kategorisering\cite{Laugwitz2008ConstructionQuestionnaire} (läs mer om ramverket och kategoriseringen under Kategorisering i sektion \ref{kategorisering}). Efter utformningen av enkätundersökningen upptäcktes det att ett komplement behövdes för att täcka frågeställningarna. Något som togs i åtanke under studiens gång var konceptet \enquote{gap-filling}. Det är en ansats som Alvesson och Sandberg \cite{Alvesson1985GENERATINGPROBLEMATIZATION} menar är när en undersökning är otillräcklig eller icke övertygande. Detta ger i sin tur upphov till obesvarade frågor eftersom att det funnits ett spektrum som inte problematiserats eller ifrågasatts. 
\newline

I studien användes två utvärderingsmetoder; enkätundersökningarna och användartest i form av fokusgrupper. Dessa metoder genomfördes sedan på slutanvändaren(målgruppen). Resonemanget till varför fokusgrupper och enkätundersökning valdes var för att få möjligheten att behandla kvantitativ data tillsammans med en kompletterande kvalitativ data. Det valdes att arbeta på ett abduktivt arbetssätt, det vill säga att vi systematiskt arbetat med en kombination av hypoteser och blandat dem med observation. 

Abduktion rymmer kunskapsfilosofiska teorier om människan, verkligheten och relationen dem emellan, men är också en forskningsmetod som bygger på att upptäcka det oväntade, stämma av det mot befintlig kunskap, skapa ny förståelse som sedan prövas genom nya iakttagelser i en levande och ständigt pågående process. \cite{Langendoen1999HowWorks}

En stor del av arbetet har varit att dra slutsatser från givna premisser, vilket är anledningen till ett akduktivt arbetssätt. Det är en av anledningarna till det valda ramverket skapat av Hassenzahl \cite{HassenzahlUserQuality} och varför kategoriseringen är central. 
\newline

\subsection{Litteraturstudie}
Vid studiens start gjordes en grundlig litteratur studie för att få nödvändig information om användbarhet och UX. Relevant litteratur togs fram i form av böcker och artiklar. I ett tidigt skede utforskades främst erkända böcker inom området, så som: 
\textit{Designing Interactive Systems by Benyon (2010)} och \textit{The design of everyday things by Norman (2002)} studerade. Den här ordningen var fundamental för att få rätt information om området. Kunskap om UX området togs fram genom grundlig studering av relevanta artiklar som exempelvis \textit{Gualiteri Best Practices In User Experience (UX) Design (2001)} och \textit{The Effect of Perceived Hedonic Quality on Product Appealingness av Marc Hassenzahl(2001)}. 
\newline

Syftet med litteraturstudien var också att finna specifik information som var nödvändig för att genomföra studien. Ett antal databaser användes för att hitta artiklar och journaler; ACM, Google Scholar, DIVA portalen och IEEE. Processen gick ut på att söka på nyckelord som kunde ge väsentliga artiklar. De nyckelord som användes vid sökningen var bland annat; \textit{Human Computer Interaction ,  Usability, User Experience,UX, HCI + Usability + User Experience, UX + Questionnaire.} 
\newline

Tillvägagångssättet som användes vid analys av källkritik på källorna genomfördes med hjälp av skolverkets riktlinjer\cite{GuideKallkritik}. Dessa riktlinjer avser att undersöka källans äkthet. De punkter som togs i åtanke vid den källkritiska analysen var\cite{GuideKallkritik}: 
\begin{itemize}
\item Finns något syfte med publiceringen?
\item Är informationen trovärdig?
\item Stämmer det med vad du redan vet?
\item Hittar du något budskap?
\item Finns dolda budskap?
\item Går fakta att kontrollera?
\item Saknas uppgifter?
\end{itemize}
Dessutom gjordes en jämförelse mellan olika källor som behandlade samma ämne, för att undersöka källans äkthet och så att det inte låg personliga budskap eller att informationen var vinklad på något sätt. Sökningen avslutades då tillräckligt med relevant information blev framtaget för att göra och forma utvärderingsmetoderna. 

\subsection{Enkätundersökning}
Enkätundersökningar är en enkel och vanlig metod för att få snabba svar samt för att få hjälpsam feedback om hur interaktionen med produkten är och hur det känns att använda den. Rent praktiskt kan det anses vara väldigt enkelt att genomföra, det är bara att skicka till ett urval av personer och man får snabba svar\cite{Paloma201499Paloma}. Däremot är det inte lika enkelt att få svar som verkligen kommer att ge värdefull feedback på en prototyp. Sådana frågeställningar kräver planering, eftertanke och noggrannhet\cite{KundKoll2018TipsKundkoll}.
\newline

Ett tydligt mål av undersökningen behövs därför vid utformningen av enkäten, hur kommunikationen ska ske, de åtgärder som kommer att göras från resultatet och uppföljningen. Frågor behöver vara ställda på ett sådant sätt att alla respondenter uppfattar dem på samma sätt vilket gör att hur enkelt det är att förstå frågan är en viktig faktor. Med en tydlig och väl utformad enkät menas därför att ge rätt och välformulerade svarsalternativ så det inte uppfattas olika beroende på respondent. En nyckel till detta är att göra olika varianter på samma fråga och testa, samt upprepa samma fråga med olika ord för att säkerställa att datan man får in är korrekt. Tydligheten påverkar också tiden som behövs läggas på en enkät där man vill hålla det så kort som möjligt för ökad deltagande, detta då respondenterna kan ha begränsad tillgång på tid \cite{Vannette20154Qualtrics}. Användaren ska dessutom kunna ge sin bedömning av produkten utan att behöva gå tillbaka och granska den igen, syftet är alltså att undvika en djupare analys vid en enkätundersökning \cite{Laugwitz2008ConstructionQuestionnaire}. 
\newline

En stor fördel med enkätundersökningar är att det kan nås ut till ett stort urval av människor då det är enkelt att skicka ut undersökningen digitalt via webben \cite{Boynton2004SelectingQuestionnaire}. Respondenterna kan också sitta i lugn och ro och svara på undersökningen vilket ger upphov till en bredare grad av reflektion innan man svarar på frågorna \cite{Boynton2004SelectingQuestionnaire}. Däremot är en klar nackdel att respondenten inte kan ställa frågor om något skulle vara oklart vilket gör att man kan misstolka en eller flera frågor. Detta kan också leda till ett visst bortfall av respondenter på grund av deras begränsade tid, vilket är beroende av tydligheten på frågorna \cite{Boynton2004SelectingQuestionnaire}. 


\subsection{Fokusgrupper}
För att angripa problemet från flera synvinklar och dessutom fördjupa resultatet från enkätundersökningen har vi valt att samla in kvalitativ data genom fokusgrupper, vilket är en systematiserad gruppintervju \cite{FokusgruppGruppintervju}. Syftet med fokusgrupper har varit att få fram användarupplevelsen i sin helhet. Vid användandet av fokusgrupper får man inte bara data om vad en grupp individer tycker och tänker, utan även information om varför. 
\newline

Den stora fördelen med fokusgrupper är att det sker en interaktion emellan deltagarna själva. Vid en intervju står intervjuaren i centrum och interaktionen är mellan intervjuare och respondent, jämfört med en fokusgrupp där deltagarna uppmuntras att dela med sig av sina synpunkter samt att lyssna på de andra deltagarna och ett av målen är att ha en givande diskussion. Morgan menar att fokusgruppers interaktion mellan deltagarna själva är lika mycket en fallgrop som styrka\cite{MorganFocusGroups}. Det som kan ske är att några deltagare dominerar diskussionen och därför är det viktigt att samtalsledaren är aktiv genom att låta alla komma till tals och styra diskussionen så den faller inom forskningsområdet och håller en röd tråd \cite{TuckmanModel}. Det finns studier som visar att fokusgrupper bör ha en samtalsledare i undersökningen, där en agerar som moderator och en som observatör \cite{Denscombe2010TheProjects}. 
\newline

Fokusgrupper är en lämplig utvärderingsmetod när man vill ha en förståelse för hur slutanvändaren resonerar eller upplever runt ett visst ämne \cite{MorganFocusGroups}. Fokusgruppens deltagare kan antingen enas, vilket kan resulteras i ett överenskommande. Alternativt så kan uppfattningen skilja sig åt, som resulterar i att det kan finnas en komplexitet runt AL1 eller enskilda delar av AL1. Vid en diskussion sker ett utbyte av åsikter där man kontinuerligt förhandlar och resonerar runt sitt ställningstagande. Vi vill i vår studie lyfta denna interaktion då den är central i vår forskningsfråga, där det kan bidra till en förbättrad slutprodukt\cite{VictoriaWibeck2010FokusgrupperUndersokningsmetod}.Genom att undersöka interaktionen mellan deltagarna och i den bemärkelsen observera och konstatera utmaningar, nöjdhetsfaktorer eller problem så kan resultat tas fram som i sin tur möjliggör att förbättringar kan ske. 
