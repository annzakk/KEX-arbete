\section{Method}
\label{sec:method}
In this section we present the method used to evaluate internal communication channels in the organization and later develop proposals that could improve the current situation. The project was divided into three main parts before the solution could be evaluated togher with RWHS. These areas where interviews (section \ref{sec:method:interviews}), background research of current situation and tools (section \ref{sec:method:research}), and lastly system prototyping (section \ref{sec:method:prototyping}).

\subsection{Interviews}
\label{sec:method:interviews}
Interviews were conducted with active members of the organization to better understand the problem, as well as what was expected from a proposed solution. The goal of the interviews were to collect as much qualitative data as possible. One of the complaints from RWHS was that different parts of the teams were using different tools or found them difficult to understand. Because of this reason it was important to understand all the problems that the actual users had in more detail than what could be provided by a simple poll. At the beginning of the project, Kajsa Sörman introduced the current situation of the organization as a whole. This was then used as the basis for all work performed later on. From this information we decided to continue our work with the following three volunteers:
\begin{enumerate}
\item Fredric Landqvist (Tech- team and head of the organization)
\item Malin Averstad Ryd (Local coordinator of Stockholm)
\item Sara Hadfy Högström (Project Manager in Malmö)
\end{enumerate}
As all these people had different roles in the organization we used different approaches and focused on their respective roles to get maximum coverage and diversity from our answers. This meant that the questions used differed to some degree which we present in the sections below (sections \ref{sec:method:interviews:fredric}, \ref{sec:method:interviews:henrik}, and \ref{sec:method:interviews:malinandsara}).

Besides investigating the organization we simultaneously worked  with companies that could provide alternative tools to cover the needs of RWHS. This meant that we also \changeRemove{had to talk}\changeAdd{talked} to representatives such as Henrik Resare from Easit, a company that sells custom CRM systems.

\subsubsection{Fredric Landqvist}
\label{sec:method:interviews:fredric}
The \changeAdd{purpose of the} interview with Fredric Landqvist (head of the organization and tech team in Sweden) was held to \changeRemove{give}\changeAdd{get} greater insight into the CRM system RWHS used. This provided information of how the system worked, main purposes with the system and reasons behind the technical choices.

\subsubsection{Henrik Resare}
\label{sec:method:interviews:henrik}
The interview with Henrik Resare was held to research another CRM option than SuiteCRM. The purpose of this interview was to get a clearer view of the process Easit sets up for a new cooperation, and also if it was possible for Easit to accommodate the primary demands of RWHS.

\subsubsection{Malin Averstad Ryd and Sara Hadfy Högström}
\label{sec:method:interviews:malinandsara}
To get a better understanding of how the current tools were used by the volunteers Malin and Sara were interviewed. Both of them worked closely with local volunteers in their respective parts of the country and could give insight into their daily workflow. As the interviews had the same purpose, the same primary questions were used as a starting point:
\begin{enumerate}
\item What services are you currently using in your team?
\item For each service: what do you use it for? What is the workflow?
\item What are your most important tools for the work? Prioritize them.
\item What are your main annoyances in the current setup.
\item What would you remove if you could?
\item Do you have any suggestions for tools to try out instead?
\item Do you use SuiteCRM. What do you think about it?
\end{enumerate}

\subsection{Research}
\label{sec:method:research}
In parallel with interviews with the current users, we conducted research of available solutions. The purpose of this was to achieve an overview of the market for the types of systems that could provide at least the same features that RWHS are using today. The three primary research topics/questions that we collected information about were:
\begin{enumerate}
\item How similar organizations satisfy their digital needs?
\item What is the viability of creating a new custom solution within the budget requirements? 
\item What are some potential alternatives for tools that are not functioning efficiently?
\end{enumerate}
For each of the tools researched we assessed them according to the following categories
\begin{enumerate}
\item How does the jurisdiction regarding secure data handling (PUL/GDPR) limit the usage?
\item Would a switch to this tools create any significant improvement?
\item Would the gain, in efficiency as well as happiness among volunteers, stand in proportion to the extra expenses generated by choosing this option?
\end{enumerate}

% These sections are not finished!!
\subsection{System prototyping and evaluation}
\label{sec:method:prototyping}
Based on the information from market research of potential solutions we matched systems with opinions and expressed need for the different parts of the organization. After collecting and creating more complete prototypes they were evaluated according to the same criteria as the individual parts (see section \ref{sec:method:research}) as well as integrational aspects. For a successful system all criteria had to be fulfilled for each part of the system necessary while providing the same or better level of service compared to what already exists. The prototypes were later presented to our supervisors at KTH as well as RWHS for response and tweaking, according to needs as well as feasibility.





