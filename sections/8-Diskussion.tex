\section{Diskussion}
\label{sec:discussion}
I det här avsnittet diskuteras resultatet av studien. Först presenteras avgränsningarna av studien och dess inverkan på studien. Efter det diskuteras kvaliteten av resultaten följt av vad man kunde ha gjort bättre. 

\subsection{Avgränsningar av studien}
En stor begränsning av studien har varit målgruppen till utvärderingen. Då Amazing Leaders inte fastställt hur deras framtida applikation skulle vara så var det svårt att kunna definiera vilka slutanvändaren skulle vara. För att få en så bra uppfattning som möjligt var målet med studien att få en stor bredd av deltagare till utvärderingarna. Det gjordes en avgränsning där enkätundersökningen målgrupp begränsades till personer inom befintliga organisationer och företag som vill lära sig mer om EQ. Urvalet för fokusgrupperna blev istället baserat på vissa kriterier (se mer om detta i avsnitt Målgrupp \ref{malgrupp}). Den här avgränsning hade en effekt på resultatet då de som var intresserade att göra enkätundersökningens inte var så många. Det har förstått att de som är intresserade för att bli \enquote{EQ-smart} var bara en begränsad grupp där även andra kunde vara möjliga slutanvändare som kunde genomföra enkätundersökningen.
\\

En annan avgränsning var att inte utvärdera användbarhet utan endast user experience. Det gjorde att vi i den mån vi hann försökte förbättra prototypens funktionalitet men att det inte var i studiens omfång. Denna begränsning gjorde att vissa funktionaliteter inte var möjliga att evaluera och det var problematiskt att försöka förklara vad skillnaden mellan användbarhet och user experience var. Trots att detta förklarades i syftet och inbjudan till utvärderingarna är det möjligt att respondenterna inte förstod skillnad vilket skulle minska reliabiliteten av resultatet.
\\

En annan parameter som var en avgränsning för studien var valet av ramverk. Vid början av litteraturstudien förstods det snabbt att det fanns många definitioner av UX och nästan lika många ramverk. Vid val av ramverk gjordes en grundlig litteraturstudie som bestämde vilket ramverk som lämpade sig bäst för studiens mål. En aspekt som kan ha förändrat hela studiens omfattning och resultat är valet av ramverk.

\subsection{Kvalitet}
För att öka kvaliteten av den insamlade datan användes ljudinspelning för fokusgrupper, som sedan transkiberades på detaljnivå. Detta reducerar möjligheten för missförstånd vid analys av data, och gör att en kvalitet kan bli försäkrad.
\\

Utöver detta reviderades frågorna till enkätundersökningen och fokusgrupperna ett flertal gånger för att försäkra sig om att deltagarna och respondenten tolkade frågan på exakt samma sätt. Det eliminerade möjliga frågetecken och tolkningsfel vid svar av enkätundersökningen. 
\\

En aspekt som påverkat studien är antalet respondenter under både enkätundersökningar och fokusgrupper. Under arbetets korta tidslinje har respondenterna över lag varit väldigt bra, då de har valt på ett sätt så att de är i rätt målgrupp. Någon som däremot skulle kunna förbättra den totala kvaliteten på både befintligt och framtida arbete är om antalet enkät- och fokusgrupps- deltagare kunde ökas. Detta då man skulle kunna introducera nya analysområden, så som statistiskt underlag samt att kunna validera resultaten på ett mer gediget sätt. 
\\

För att kunna göra detta skulle man även behöva omformulera enkäten på ett sätt så att det skulle vara mer lämpad för djupare analys. 
\\

\subsection{Förbättringsområde}
En del av arbetet tillsammans med uppdragsgivaren var att etablera målgrupp och avgränsa arbetet. Om denna tid kunde allokeras till annat hade det varit fördelaktigt för studien i sin helhet då det skulle öppna upp tidsplaneringen. \\

Det hade varit bra om man ser igenom enkäten för att se hur man kan förändra och förbättra den för att extrahera mer information. Det är svårt att göra en bra analys då den data man får in är en skala från 1-6. Man skulle till exempel kunna be respondenterna rangordna kategorierna. Då skulle man kombinatoriskt kunna se vad folk tycker är viktigast.\\

Studien använde fokusgrupper för att få kompletterande och komplett data. Ett förbättringsområde även där hade varit att se över genomförandet av fokusgrupper för att få ut mer ur den timmen som var avsatt för diskussion i fokusgruppen. 