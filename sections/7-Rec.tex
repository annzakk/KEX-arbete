\section{Generell rekommendation och sammanställning}

\subsection{Vad vi kan rekommendera som nästa riktning}
Från resultatet ser man att de olika kategorierna har relativt varierande svar. Det är överlag höga betyg för kategorierna. Rangordningen var
% Som!man!kan!se!i!diagrammen!får!de!olika!kategorierna!relativt!enhetliga!svar.!Det!är!
% överlag!höga!betyg!för!samtliga!kategorier!men!man!kan!också!se!skillnader!mellan!de!
% hedonistiska!–!och!användbarhetskategorierna.!För!de!hedonistiska!kategorierna!ligger!
% antal!svar!på!fem!(5)!och!sex!(6)!på!totalt!43%!respektive!53%,!medan!de!för!
% användbarhetskategorierna!ligger!på!70%,!61%!respektive!64%.!Detta!ser!vi!som!en!
% indikator!på!att!användbarheten!för!systemet!bedöms!som!bättre,!av!de!svarande,!än!de!
% hedonistiska!kategorierna!gör.!

\subsection{Resultat från utvärdering av prototyp}
Överlag hade användaren en positiv bild av AL1. Vad fokusgrupperna diskuterade
\\
% se: http://www.csc.kth.se/utbildning/kandidatexjobb/teknikmanagement/2010/rapport/jensen_anna_OCH_persson_johan_K10052.pdf

Datan visar på att tydlighet är viktigast. På prototypen ansåg de flesta att den var bra. Trots det var det 4 och 12 procent som tyckte den var svår att förstå och komplicerat. 
Från fokusgrupperna förstod man att medlemmarna inte tyckte det var \enquote{glasklart} vart man skulle markera. 
\\
Till en början rekommenderar vi att man 

När man då tittar på vad fokusgrupperna sa var