% Present your results/findings.
% Make sure that your results meet the purpose and goal of the project.
% Be sure to distinguish between Results (what you found out/achieved) and Discussion.
% (Discussion is included in the next section)

% THIS IS WHAT WE SAID IN THE METHOD
% The three primary research topics/questions that we collected information about was:
% 
% How similar organizations satisfy their digital needs?
% What is the viability of creating a new custom solution within the budget requirements? 
% What are some potential alternatives for tools that are not functioning efficiently?
%
% For each of the tools researched we assessed it according to the following categories
%  How does the jurisdiction regarding secure data handling (PUL/GDPR) limits the usage?
%  Would a switch to this tools create any significant improvement?
%  Would the gain, in efficiency as well as happyness among volunteers, stand in proportion to the extra expences generated by choosing this option?

\section{Results}
In this section the results and findings from the investigation are presented. The method for conducting and evaluating our research are found above, in section \ref{sec:method}. First we briefly introduce the answers from interviews with peoples in different positions in the organization. After that we present our proposed solutions and how well they fit according to defined evaluation criteria.

% This is the table that contains all cost proposals
\begin{table*}[!t]
\centering
\caption{Sammanfattning av resultat}
\label{tab:cost-proposals}
\begin{tabularx}{\linewidth}{XXXXXXX} % llllll
\toprule \\
& Fokusgrupp 1    & Fokusgrupp 2           & Fokusgrupp 3                                                  & ----               & ---         &                                          \\
\toprule \\
Nöjda med den grafiska designen av hemsidan   		& Partially                           & No                                                               & Yes                           & Yes                                                   & Yes                                      \\
\midrule \\
Var intresserade av att klicka vidare och förstå mer     	& Partially                           & Yes                                                              & Yes                           & Yes                                                   & Yes                                      \\
\midrule \\
Tyckte det fanns för mycket funktionaliteter/verktyg     & 4 +                                 & 4 +                                                              & 2 +                           & 1                                                     & 4 +                                      \\
\midrule \\
---- & 0                                   & 0                                                                & \textgreater 50000 SEK         & \textgreater 50000 SEK                                 & 0                                        \\
\midrule \\
Monthly fee         & 0                                   & 0                                                                & 1000 SEK                       & \textless 200 SEK                                      & $\sim$ 35 SEK /user                       \\
\midrule \\
Pros                & No change required                  & Familiar to most users, all services needed included, quite simple & Efficient/fit for purpose     & Very Efficient, everything could be integrated        & Familiar to most users, simple interfaces \\
\midrule \\
Cons                & Complicated, not compliant with laws & Many different tools are needed                                  & Expensive, new system to learn & Expensive, new system to learn, requires hiring someone & Many different tools are needed \\  
\bottomrule
\end{tabularx}
\end{table*}

\subsection{Interviews}
In this section we briefly introduce the answers given by each of the interviewees. More detailed transcripts or summaries can be found in the appendices. 

\subsubsection{Fredric Landqvist}
\label{sec:method:fredrik-landqvis}

Fredric explained that one of the primary reasons for the choice of SuiteCRM was that users can work with relationships between the parties, yet also because RWHS could internally handle personal data whilst still complying with current regulations.

Regarding the problems with today's solution, Fredric addressed that the greatest disadvantage of SuiteCRM is that it is too big, in the form of presenting to much functionality to users. A lighter version of it would be better, as it would be easier for new and non technical users to use. 

\subsubsection{Henrik Resare}
% The interview with  Henrik Resare (Sales Manager at Easit)  gave a more clear view of the process Easit sets up for a new cooperation and how they can meet Refugees Welcome demands. 

Henrik explained that Easit takes care of all the technical logic that has to be developed and that it would be no problem for Easit’s solution to interact with RWHS's web page or Wordpress. Furthermore he told us that Easit works with various types of organizations and that they can always make customized workflows depending on the organizations demands. Regarding security and the PUL law, Easit stores all data in Sweden and already have functionality for the new updated GDPR law. 

\subsubsection{Malin Averstad Ryd}
Malin explained that she is now used to having many tools, but it would be better if they were fewer. Malin also reflected on that she is the only member in Stockholm using Slack, since the volunteers are only interested in matching hosts with tenants, and due to limitations such as age and lack of smart phones with volunteers. She does however inform her team about the discussions that are conducted on Slack, using emails, text messages, and when a face to face contacts occurs.

Malin also talked about the importance of Google Drive to refer to documents, as well as that the CRM that RWHS are using is not functioning properly as it should.
% like how other CRMs used by RWH. 

\subsubsection{Sara Hadfy Högström}
Sara Hadfy Högström explained in her interview that she is relatively satisfied with the workflow in her region but that she probably uses too many services herself (around 12 different).  The full transcript is available in Appendix \ref{app:sara-hadfy} but a short summary is presented below.

She felt that the workload was decreasing and that it would be hard to justify an investment in new and expensive systems. As the future financing of the organization is uncertain an extra monthly fee would also be hard to motivate. Her main complaints of the current situation were regarding SuiteCRM being hard to learn, having technical issues and not always working as intended.

The team that she was working with were at the time using WhatsApp for daily communication and Trello for handling the landlords. Data was periodically transferred to Trello from SuiteCRM because Trello was more efficient, gave a better overview and was easier for collaboration. She could agree on switching to G Suite instead of Trello if needed however expressed that it would be unreasonable to work solely on SuiteCRM.

% This is what we promised to provide in form of background research
% How similar organizations satisfy their digital needs?
% What is the viability of creating a new custom solution within the budget requirements? 
% What are some potential alternatives for tools that are not functioning efficiently?

% For each of the tools researched we assessed it according to the following categories
% How does the jurisdiction regarding secure data handling (PUL/GDPR) limits the usage?
% Would a switch to this tools create any significant improvement?
% Would the gain, in efficiency as well as happyness among volunteers, stand in proportion to the extra expences generated by choosing this option?

\subsection{Solutions on the market}
In this section, we present the findings about each of the different alternative solutions that we found for RWHS. A summary of the primary aspects of interest are presented in table \ref{tab:cost-proposals}.

\subsubsection{Current Solution}
\label{sec:result:current-solution}
% As the the research was gathered an understanding was made that the current solution was not legal. The current solution did not follow the PUL law since it was gathering some information about people in an structured way in Google Drive. A change is therefore needed, especially with the GDPR law in mind. 
% The research and interviews showed that the current CRM is very “messy” and is not contributing in the sense of how technology and ICT solution should contribute to efficiency in daily work. 
The current solution, which consists of a combination of different tools, has problems with both legal issues as well as usability. This means that it is not an optimal choice and some changes are necessary immediately to avoid fines. It should be a priority to conduct the following three changes:
\begin{enumerate}
\item Stop using Trello as it does not comply with laws regarding personal data (PUL/GDPR) (see section \ref{sec:background:trello}).
\item Make sure that documents on Google Drive are only shared with users that require them, to make sure that it complies to PUL and GDPR. This includes both read and write access.
\item Ensure that RWHS comply with GDPR in all ways before 25th of May 2018.
\end{enumerate}
To further improve the current system we found most users to agree that the following changes would improve the experience with the system.
\begin{enumerate}
\item Clean interface and remove unnecessary functionality
\item Reduce number of data items visible during regular use. In the default mode, only contact information related to relevant registrations should be visible.
\item Make sure that the system runs quicker and has fewer technical problems/bugs.
\item Add easier ways of following the development of the situations with landlords in each region. 
\end{enumerate}

% After interviews and gathered information we saw as a result that the current solution needs to be improved: 
% We have concluded this in mainly three different areas that is presented as a result if current solution should stand. These points should be improved.
% \begin{enumerate}
% \item Clean current platform (CRM)
% \item Invest in education. Better documentation
% \item Remove Trello since it is not legally aligned with GDPR
% \item Make sure to follow the GDPR law.
% \end{enumerate}

% Furthermore we have concluded that since the current solution is not legal it shall be replaced and these are the other options that we would like to present as solutions. 

\subsubsection{Easit and other CRMs}
A new CRM with a less cluttered interface and functionality that is customized to RWHS demands is optimal. We found multiple different providers offering this service however all which of them were expensive (above 50 000 SEK). We decided too look into one of the options called Easit. This solution could still be an option as investing in the system has a lot of advantages. One advantage is that Easit takes care of all the functionality, support and education. They have also made it clear that the integration from RWHS's web page to Easit's systems will not be a problem. These were the two greatest concerns and challenges Fredric Landqvist addressed, provided that RWHS would change system (more information about his interview can be found in section \ref{sec:method:fredrik-landqvis}). Easit's system is available in two versions, as a service with a monthly fee or as product that a organization could buy. In the second case only a smaller fee for support and maintenance would be payed each month.

% Easit offers two types of solutions, one solution is renting their services and systems. By renting, RWHS have to pay a monthly fee. The other solution is to buy Easit's services and systems, doing this a higher one time cost have to be payed but then a much lower monthly fee is payed for support and maintenance. 

% This belongs in the discussion
% This could be a problem because as said RWH is a non-profit organization.
% However this solution comes with a price, integrating RWHS:s data and making an external system compatible with RWHS:s web page could be expensive. 
% If RWHS could apply for some kind of economic contribution, buying the solution is a better option as we think the monthly payment for this is not as significantly as it would have been if the organization decides to rent.      

\subsubsection{G Suite}
\label{result:google-suite}
% On the 25th May of 2018 the European Union will implement the GDPR. This law make it possible for companies outside of EU to contribute as long as they are following the GDPR law. The law will strengthen the rights that individuals have and unify all EU residents. Regardless of where the data is processed the GDPR law seeks to unify the data protection for all EU residents. Therefore it is possible for companies outside of EU to handle data about individuals, and have their servers in “third-countries” as long as they are following GDPR.  


The collection of G Suite services could provide RWHS with all necessary tools for conducting daily tasks, including collecting new contact information where the CRM previously was used. This could be done using the tool \textit{Google Forms} which automatically collects user data into spreadsheets which would be available to the volunteers through \textit{Google Drive}. In addition to providing the necessary tools, all of them also complies with the new GDPR standard as presented in section \ref{sec:background:g-suite}. A switch to this tool chain would simplify the daily workflow and reduce the total number of tools used in different parts of the country. At the same time the tools would be familiar to most users and not provide unnecessarily complex interfaces. A negative aspect is how data is stored. As all registrations would be kept in spreadsheets, it could be \changeRemove{difficile}\changeAdd{difficult} to search through the data and store it longer periods.

% The only thing needed is an account providing the domain name that the organization want to use with Google services. Once this is done the services are available, which includes Gmail, Calendar, Drive, and other core G Suite services. 

% For RWHS Google offers a free Google Suite for nonprofit organization. Refugees Welcome would eligible, but needs as a first step do an application: 

% Our recommendation is that in the website you simply do a google form, where you register and then the registration will automatically go into a google drive. This step means you will skip CRM and make a simplified work flow.

\subsubsection{Custom Solution}
A perfect solution does not exist on the market. A perfect solution would therefore be a custom solution specifically produced for RWHS. A custom system would be customized for RWHS's specific requirements, their workflow and the data handled. It would be an expensive solution to create, therefore one option is to start from an already existing system (suiteCRM) and configure the system to better fit  RWHS's needs, which is discussed in section \ref{sec:result:current-solution}.

% The workflow for RWHS is very specialized and does not really fit into a category where tools already exists on the market for their specific purpose. This means that the best would be to create a custom tool for handling data in the way that the organization is currently doing. Such a system could include all information and support for the workflow. This type of system would be expensive to create and one option is to start from an already existing system (suiteCRM) and configure it to fit better with the needs which is discussed in section \ref{sec:result:current-solution}.

\subsubsection{Microsoft Office 365 Enterprise (for non-profits)}
One alternative to G Suite is the office package offered by Microsoft. It offers approximately the same functionality, whilst offering more recognizability with users. What is lacking is the option to integrate a form creation tool, which exists in G Suite with Google Forms. Microsoft, as Google, also offer their system with a reduced cost for non-profit organizations. One feature from Microsoft that is absent in Google Suite is a web app called Microsoft Tasks which could be a Trello replacement that still follows GDPR as presented in section \ref{sec:background:microsoft-office}. 



