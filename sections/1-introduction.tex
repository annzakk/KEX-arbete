% Describe the background to the problem and make a clear problem formulation. Make
% sure the reader understands the context and why the question is interesting from both
% an ICT and sustainability perspective.
% The sustainability aspects must be clearly evident from the problem formulation.

% Problem description: Which problem does the project attempt to solve, why is this in-
% interesting from a sustainability perspective?

\section{Introduction}
\IEEEPARstart{R}{efugees Welcome Housing Sweden} is a non profit organization with the aim of connecting landlords in Sweden with refugees by matching them on a personal basis \cite{Welcome}. If a potential landlord is interested in renting an apartment, letting or subletting their apartment they can easily visit the web page that is run by the organization and fill out their contact information. One key point, and the main purpose of RWHS (\textit{Refugees Welcome Housing Sweden}), is to provide refugees with housing that is cheap enough for them to afford, by providing a personal relationship between them and the landlord. 

This processes of finding and matching landlords and tenants is in the current state not as optimal as it could be. The main issues are related to technical problems with handling the complex systems that are used in the organization. Volunteers use multiple different services to cover their digital needs which is undesirable. A simpler, more complete and integrated system that is easy to understand and get started with, even for non technical users is necessary to enable greater accessibility for everyone \changeAdd{working in the organization}. A solution to this problem would increase the productivity and efficiency for the organization as a whole. An updated system would need to be able to handle personal and sensitive data from refugees as well as landlords in a way that is both easier and legal according to current regulations. It is not necessary for all parts to be integrated, however ease of use is a primary objective and a fragmented system was identified as a primary hindrance for usability.

In this report the current state of the organization were assessed, and the the needs for the different parts of the daily operation were identified. Based on this a proposal for alternative systems were created, that could help improve the workflow and efficiency without compromising security and the strict economic boundaries that were connected to this case.

% In this report we try to assess the current state of the organization and identify needs for different parts of the daily operation. Based on this we will create a proposal for alternative systems that could help improve the workflow and efficiency without compromising security and the strict economic boundaries that are connected to this case.

\subsection{Problem}
\label{sec:problem-definition}
After an initial study of the organization, three key problem points could be identified which this report investigated.
\begin{enumerate}
\item \textit{Over separation of internal and external information sharing, caused by a fragmented system of services.} Easy communication, both internal and external, is needed to organize workers and inform them about procedures and regulations. The current solution stores and communicates information across a multitude of platforms which are hard to synchronize. 
\item \textit{Usage problems for new as well as more experienced volunteers.} The volunteers are required to learn a multitude of services, just for get started. This time is lost to technical details instead of starting on the important work. It disconnects the volunteers from working at the major issue. Beyond being an extra time waste, it creates a hindrance for volunteers and decreases morale within the organization as the current procedure is way too inefficient.
\item \textit{Secure handling of personal data.} RWHS handles personal and sensitive data from refugees, as well as landlords. Substantial parts of this information is lawfully required to be securely kept, which has to be considered when proposing a digital solution.
\end{enumerate}

\subsection{Sustainability}
\label{sec:introduction:sustainability}
RWHS as an organization is in the current state trying to tackle a problem, that even with large resources, would be complex and hard to solve in a good way. At the same time their organization is completely dependent on the willingness and passion from volunteers as well as donations to be able to continue operating. Whilst RWHS are in a limited situation economically, they are providing newly arrived \changeRemove{families}\changeAdd{refugees} with more personal relationships and better treatments than elsewhere. This also means that the \changeRemove{families}\changeAdd{refugees} have a larger chance of successfully establishing themselves in Sweden. Not only does this put the families in a better place for building a new and successful life in Sweden, yet it also eases their integration into the Swedish society which is important for creating an accepting and socially sustainable environment. This type of work improves upon the condition of UN's SDGs
(\textit{Sustainable Development Goals)} \cite{UnitedNations} such as: 
\begin{itemize}
\item Goal 10 - Reduced Inequalities
\item Goal 11 - Sustainable cities and communities 
\item Goal 12 - Responsible consumption and production
 \item Goal 16 - Peace, justice and strong institutions
\end{itemize}
More information about the goals and their connections to RWHS are found in section \ref{sec:background:sdg} and \ref{sec:discussion-sdg} (\textit{\nameref{sec:background:sdg}}).

Because of the economical limitations of a non profit organization it is sometimes difficult to acquire adequate knowledge and funds to improve IT systems, which is a struggle that commercial companies do not experience to the same extent. If a replacement of the current system for a better alternative is possible it would contribute to an increase of time efficiency and decrease the risk of volunteers dropping of. This means that more energy could be spent on the tasks that actually matters for the volunteers. This implies that our work as an unbiased source of information for the organization indirectly helps to achieve a more sustainable society. 