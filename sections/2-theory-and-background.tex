\section{Theory and background}
\label{sec:background}
In this section we discuss the different tools and background that is needed to understand the current and possible future solutions in more detail. We first develop on the concept of PUL and GDPR with restrictions in handling personal data (section \ref{sec:pul} and \ref{sec:gdpr}). After that we briefly introduce the tools that are used within the organization today (section \ref{sec:current-tools}).

\subsection{PUL / PDA}
\label{sec:pul}
The Personal Data Act (PUL) is a Swedish law that exists to protect people and their personal integrity.  In general the law covers how personal documents should be handled to be safe as well as prevent unauthorized people from accessing and misusing them.  One important aspect here is that information should not be available to people that don't need it.
% It protects all records that counts as personal data and limits how they can be handled. Redundant (Julius)
Personal data or information is a wide term here that includes everything which could potentially be used to identify an individual \cite{Datainspektionen2015}. The full definition from Datainspektionen in Swedish is available in appendix \ref{app:personal-information}.

According to the definition we call anyone using a cloud service for processing personal data \textit{the controller} of the personal data, even if the processing is carried out by a cloud service provider or its subcontractors. PUL forces any controller of personal data, who also makes use of cloud services, to ensure that a new third party cloud provider does not violate the law and that the following criteria are met \cite{DATAINSPEKTIONEN} :
\begin{enumerate}
\item Ensure that there are no risks that the personal data may be used for other purposes than the original one,
\item Ensure that the personal data saved in the cloud service can’t be available to a third country, i.e in a country outside of EU,
% But if stored in a country outside, it is supporting the Personal Data Act.
\item Assess what security measures has to be taken in order to protect the personal data that is processed,
\item Ensure that a processor agreement is drawn up with the cloud provider, and also
\item Consider other legislation such as confidentiality legislation.
\end{enumerate}

\subsection{GDPR}
\label{sec:gdpr}
From the 25th of may, a new law called GDPR (\textit{General Data Protection Regulation}), will replace PUL. The GDPR has been constituted by the EU (\textit{European Union}) and will replace data protection laws in all countries where GDPR is active. The current legislation for many countries  were enacted before cloud services and Internet where widespread, which is a reason that the previous laws are a bit outdated. The GDPR is supposed to strengthen data security and improve trust in the digital era. At the same time it will benefit business by unifying the laws on the European market which simplifies legal processes.

It is relevant to take into consideration that when the laws are changed, GDPR will apply to any company that are dealing with data that belongs to residents in the EU. This means that it will be possible for companies to provide services for handling personal information, even if they host their data outside of EU. It introduces better possibilities for transferring data and makes it economically possible to create compliant systems because of the reduced differences between different countries \cite{Google}. Another big change is that if you fail to follow GDPR, the penalties can be as large as 20 000 000 Euro or 4\% of the conglomerate global revenue which makes it key to follow \cite{MULTISOFT}. This can be compared to PUL where penalties have been both less common and severe.

Work on the GDPR started 2016, however the EU agreed with the final text that all organizations and businesses has to comply with the GDPR by the 25th of May 2018. Similar to PUL, both controllers and processors of data need to comply with the GDPR. The controller would in this case be RWHS, whereas the processor would be the company actually handling the data processing. \cite{EuropeanComission2016}

\subsection{Tools}
\label{sec:current-tools}
The organization is as previously mentioned using a multitude of different tools to fulfill their different needs. In this section we will briefly introduce some of the services that are used today: SuiteCRM, Trello, G Suite, Slack, as well as a few other daily communication tools. Some prospective candidate tools are also presented as Microsoft 365 Enterprise.

\subsubsection{SuiteCRM}
SuiteCRM is a software fork of the CRM (\textit{customer relationship management}) system SugarCRM. SuiteCRM was released October 2013 and is a free open source alternative to the more complex SugarCRM. RWHS are using SuiteCRM for tracking all relationships and contacts with landlords and tenants. It is also used as a tool for the volunteers to keep track of the states of the different application processes. \cite{CommunitySuiteCRM2014AboutDocumentation}

\subsubsection{Trello}
\label{sec:background:trello}
Trello is an on-line tool for managing projects and progress. It can be shared among a group to organize projects in a task based form \cite{Trello}. RWHS are using Trello to assign tasks to volunteers and control the progress of matching hosts with tenants. According to their own documentation, Trello is not following the laws (as presented in section \ref{sec:pul} and \ref{sec:gdpr}) regarding secure handling of personal data \cite{Trelloa}. 

\subsubsection{G Suite (Gmail, Google Drive, and Google Hangouts)}
\label{sec:background:g-suite}
G Suite is a complete office suite from Google that includes services such as word processors, spreadsheets, emails, forms with data collection as well as file storage. The tools are available for individuals and organizations. Today this is the primary office suite for RWHS and all previously mentioned features are used on a regular basis. \cite{GoogleAnvandarvillkorSuite}

Google has made a clear statement that they will follow GDPR, which opens up for the possibility to use G Suite, both in the free form but also the enterprise version which is available for free to non-profit organizations \cite{Googlea}. Google also commits to help their customers with their GDPR compliance journey by providing security and privacy protections built into all their services and even offers to sign a contract ensuring this. RWHS has already applied for the non-profit program which they got granted.

\subsubsection{Slack}
Slack is a work communication platform that provides teams with quick and organized text based messages. Slack provides a single chat room for the entire organization where individual chat as well as group chat is provided. RWHS are using Slack for communication on a local as well as national level. \cite{SlackAboutSlack}

\subsubsection{Other communication tools}
RWHS has used Slack as their primary daily communication tool however volunteers often choose another communication tool based on previous experience and preference. Both Facebook, Whatsapp, and Hangouts are regularly used by different members.

\subsubsection{Microsoft Office 365 Enterprise}
\label{sec:background:microsoft-office}
Microsoft has a complete office package that includes all basic tools for communication and recording information. These includes things such as email clients, word processors and spread sheets. It also includes the tool Microsoft Tasks which could replace Trello. \cite{MicrosoftMicrosoftNonprofits}

\subsection{Sustainability}
\label{sec:background:sustainability}
This chapter introduces relevant sustainability aspects, that will later be discussed in section \ref{sec:discussion}. Section \ref{sec:background:social-sustainability} introduces social sustainability aspects that are relevant to RWHS's work, section \ref{sec:background:sharing-economy} introduces the concept of sharing economy, and section \ref{sec:background:sdg} explains the UN's SDGs whilst focusing on the four most relevant goals.

\subsubsection{Social sustainability}
\label{sec:background:social-sustainability}
The definition of social sustainability has been widely discussed. One definition developed by Social Life is  \textit{"a process for creating sustainable, successful places that promote wellbeing, by understanding what people need from the places they live and work. Social sustainability combines design of the physical realm with design of the social world – infrastructure to support social and cultural life, social amenities, systems for citizen engagement and space for people and places to evolve."} \cite{Woodcraf}

A component of social sustainability that Social Life presents is bridge sustainability, which they describe as changing behavior to achieve bio-physical environmental goals. \cite{Vallance2011}

\subsubsection{Sharing Economy}
\label{sec:background:sharing-economy}
Sharing Economy is an economic term used to describe a system in which assets or services are shared between private individuals, either for free or for a fee. \cite{Albinsson2018TheEconomy}

%% We already said that these to sections discuss how we indirectly affect
%% sustainability factors. It is redundant information to add this again.
% If RWHS implements one of the provided solutions this project will have an indirect effect on the \textit{Sharing Economy} perspective. By making the required tools and the work flow more simple, the volunteers can work more effective and as a result match more landlords with refugees.

%% Yes, as previously mentioned... We don't need to say this again.
% As mentioned the solutions this project offers  only have an indirect effect on this aspect and there is no guarantee that more landlords and refugees will be matched based only on providing more effective digital tools for RWHS.

\subsubsection{Sustainable Development Goals}
\label{sec:background:sdg}
UN has a collection of 17 goals for sustainability. This report focuses on four goals in more depth.

\textit{Goal 10 - Reduced Inequalities} aims to reduce social, economic and political inequalities within and among countries, regardless of age, sex, disability, race, ethnicity, origin, religion or economic status. 
\cite{UnitedNationsb}

\textit{Goal 11 - Sustainable cities and communities} is about making cities and human settlements safe, inclusive, sustainable and resilient. It is relevant to consider that Stockholm has many young adults and there is a current housing crisis. \cite{Sheiban2002Den1800-talet} \cite{UnitedNationsb}

\textit{Goal 12 - Responsible consumption and production} is about forming long term patterns that will be integrated with national and sectoral plans, companies and consumer behavior. \cite{UnitedNationsGoalPlatform}

\textit{Goal 16 - Peace, justice and strong institutions} described by UN: \textit{"Promote peaceful and inclusive societies for sustainable development, provide access to justice for all and build effective, accountable and inclusive institutions at all levels"}. This goal also includes working against physical and sexual abuse of individuals. \cite{UnitedNationGoalPlatform} 
 

