\section{Discussion and Conclusion}
\label{sec:discussion}
In this chapter we present the conclusion from the project, and discuss the method and result presented in previous chapters. We also reflect on  the projects limitations, presented solutions and the implications this project can have on sustainability factors. 

\subsection{Limitations}
During this project there were a set of limitations to consider. The project contained two different types of limitations, school limitations and organizational limitations. Because this mainly was a school project we had to adapt our method and result to fit the timescale and goals within the course. As RWHS is a non-profit organization we had strict economical limitations as well.

Due to these restrictions there is a chance that our result is not optimal for other organizations and due to the time limit there could be other solutions on the market that has not been researched and analyzed. Most non free solutions were simply not considered.

\subsection{Pros and cons of solutions}
Using only productivity as a factor, using Easit or another costly CRM would be the best solution. Easit would be an easily integrated solution which would require little modification of the current work flow. Unfortunately cost is a big restriction, which eliminates Easit as an alternative.

Keeping the current solution, with small alterations to ensure compliance with GDPR as well as reorganizing the current CRM, would be cheap both considering money and effort. It would however not completely solve the problem regarding the usability of SuiteCRM. It would only slightly reduce the usability problems, however not require a great adjustment of work flow, which is an advantage.
%A big advantage would be as previously mentioned, low cost in both time and effort. Additionally it would not require great adjustment of work flow.

To use only G Suite would be a slightly bigger change, however all of the problems regarding CRM usability would be mostly solved. It would require minimal effort to change the current registration link from the CRM form to a Google Form and have the data collected into spreadsheets automatically. Some regions that today use Trello for landlord management would have to switch to using a Google Sheet, however the regions that already use G Suite for landlord management would instead automatize their work flow, by eliminating the CRM as a middle step, and then manually moving the data to Google Drive.

A switch to Microsoft's office suite does not feel motivated as it would increase the costs of the organization, yet not add much new functionality. At the same time there is a new system to implement and learn which would not be optimal.

\subsection{Sustainability}
Achieving sustainable growth requires changes in many traditional industrial processes. The technology should make everyday work easier in companies and organizations \cite{Vergragt}. In RWHS an automation of the tools would contribute to a more efficient workflow and more motivation among the volunteers. This would hopefully result in more work towards the goals of the organization, which addresses very relevant sustainability factors in our society today. By contributing to the work environment for RWHS we hope to indirectly affect the same sustainability goals that they are directly affecting.

As we only have an indirect effect on this aspect, there are no guarantees that a greater number of landlords and refugees would be matched based only on providing more effective digital tools for RWHS. On the other hand, having good tools is\changeRemove{ really} a prerequisite for performing a good job, which \changeRemove{means that an upgraded system would at least maximize}\changeAdd{would suggest that an upgraded system could maximize} their abilities to do what they are doing as good as possible.

This chapter discusses some of the most important sustainability factors that RWHS are contributing to. The focus is social sustainability, sharing economy and UN's sustainability goals which are discussed in sections \ref{sec:discussion:social-sustainability}, \ref{sec:discussion:sharing-economy} and \ref{sec:discussion-sdg} respectively.

% This is already in the report in the problem section
% \textbf{Are we successfully investigating and solving our problems?}
% \textbf {Over separation of internal and external information sharing, caused by a fragmented system of services. Easy communication, both internal and external, is needed to organize workers and inform them about procedures and regulations. The current solution stores and communicates information across a multitude of platforms which are hard to synchronize. 
% Usage problems for new as well as more experienced volunteers. The volunteers are required to learn a multitude of services, just for get started. This time is lost to technical details instead of starting on the important work. It disconnects the volunteers from working at the major issue. Beyond being an extra time waste, it creates a hindrance for volunteers and decreases morale within the organization as the current procedure is way too inefficient.
% Secure handling of personal data. RWHS handles personal and sensitive data from refugees, as well as landlords. Substantial parts of this information is lawfully required to be securely kept, which has to be considered when proposing a digital solution.}

\subsubsection{Social sustainability}
\label{sec:discussion:social-sustainability}
Since RWHS is a non-profit organization and the volunteers are working without pay, the aim with the technology should be to optimize the volunteers work and make it as interesting as possible. Internally the tools should contribute to efficient work so that the volunteers can focus on work tasks that are fun and where they feel that they contribute in an effective way. This would improve their work quality and in a long term perspective give a work flow that could include more volunteers and make them perform better.

The definition of social sustainability has been widely discussed but the one we introduced in section \ref{sec:background:sustainability} (\textit{\nameref{sec:background:sustainability}}) is aligned with RWHS's work. RWHS are finding housing for refugees who do not know anyone in the Swedish society yet, which contributes to the refugees situation by giving them an improved social and culture life that only a personal connection in a new place will give you. At the same time \changeAdd{some of} the refugees are \changeRemove{mostly}\changeAdd{, according to Kajsa Sörman,} using preexisting resources (spare rooms), which improves the efficiency at which heated space is used.

The social sustainability contributions that RWHS are doing could also be classified as bridge sustainability. RWHS with their platform encourages potential landlords to give up otherwise unused space, and by doing that helps the potential landlords to contribute to the bio-physical environment as a consequence. This is a non-transformative approach as it doesn't require changes in either life style or believes.
 
\subsubsection{Sharing Economy}
\label{sec:discussion:sharing-economy}
By creating new opportunities to create closer relationships between tenants and landlords, RWHS is also creating a rich climate for a sharing economy where parts of apartments could be shared if single rooms are rented by refugees. From a sustainable point of view it is positive for people and communities to share some of their resources. One example is household appliances such as ovens, laundry machines etc. that are only used a fraction of the time. Instead of producing machines for every individual it is more efficient to share them among many.

%% We already said that these to sections discuss how we indirectly affect
%% sustainability factors. It is redundant information to add this again.
% If RWHS implements one of the provided solutions this project will have an indirect effect on the \textit{Sharing Economy} perspective. By making the required tools and the work flow more simple, the volunteers can work more effective and as a result match more landlords with refugees.

%% Yes, as previously mentioned... We don't need to say this again.
% As mentioned the solutions this project offers  only have an indirect effect on this aspect and there is no guarantee that more landlords and refugees will be matched based only on providing more effective digital tools for RWHS.

\subsubsection{Sustainable Development Goals}
\label{sec:discussion-sdg}
As previously mentioned in the background, the most relevant of UN's SDGs in this specific case are:
\begin{itemize}
\item Goal 10 - Reduced Inequalities
\item Goal 11 - Sustainable cities and communities 
\item Goal 12 - Responsible consumption and production
\item Goal 16 - Peace, justice and strong institutions
\end{itemize}
We introduced these goals since we believe these are some of the goals that RWHS work is aligned with. We will present all of the goals and why RWHS work is widely important. 

\textit{Goal 10 - Reduced Inequalities} 
RWHS main goal is to connect newcomers with hosts that offers housing. This matching supports the integration of newcomers with the Swedish society. Such integration could also develop the social aspects of the refugees towards the society, by understanding the culture and traditions. This hopefully makes them feel more equal and connected with other members of the Swedish society.

Beside the social inclusion, RWHS helps with economic inclusion as well. By quickly providing newcomers accommodation, this helps them settle and allows them to focus on having better chances of employment and higher income, which contribute to having equal economy, comparing to other members in the society.

\textit{Goal 11 - Sustainable cities and communities} 
RWHS is in alignment with goal 11 since they enable more efficient use of available housing. For instance, if all young adults had an apartment it would not be sustainable. As section \ref{sec:discussion:sharing-economy} brings forward, sharing resources is necessary to have sustainable cities and communities. 
\cite{UnitedNationsb}

\textit{Goal 12 - Responsible consumption and production}, by matching together people and having them share housing and resources the organization consequently works toward this goal. This SDG and the impact RWHS has on it can be related to the aspects that section \ref{sec:discussion:sharing-economy} (\textit{\nameref{sec:discussion:sharing-economy}}) brings forward.

\textit{Goal 16 - Peace, justice and strong institutions}. As mentioned in section \ref{sec:background:sdg} (\textit{\nameref{sec:background:sdg}}) this goal includes working against physical and sexual abuse of individuals. In 2016 \textit{Kalla fakta}, one of Swedens biggest investigative journalism programs revealed a series of serious abuse of refugees in a refugee accommodation in Dalarna \cite{TV4KallaTv4.se}. The same year Expressen revealed similar events taking place on another accommodation \cite{RoosAsylrapporten:Asylboendet}. News of this variety occur on a periodically basis, both about refugees and in other types of accommodations where vulnerable individuals are located. The Swedish Migration Agency has been criticized for not controlling and taken care of these kinds of problems. To prevent problems of this kind, RWHS interviews the landlords to detect any possible warning signs. RWHS also value the relationships between the refugees and the landlords very highly. By providing RWHS with a more effective tool they can focus on these parts more and hopefully prevent abuse in the homes refugees end up living in.  
 
\subsubsection{Goal conflicts}
As the result has shown, RWHS follows several of the SDG, however there are goal conflict that could occur such as a conflict between Goal 10 and Goal 12. We have by now understood RWHS's work and what they do. If we only respond to their sharing resources as goal 12 - Responsible consumption and production their work for helping refugees could be seen as only helping one minority. This is a widely discussed topic for those who feel that 'favoring' a minority in Sweden is not right, where there is a housing crisis. There are students in Sweden that can not find apartments and RWHS is giving a service to find housing for a specific minority. Concluding this could unfortunately be a goal conflict with goal 10.

\subsection{Final recommendation}
Considering both requirements and limitations, our final recommendations are that RWHS should develop a standard for how the workflow looks in all teams, and make sure that this standard is compliant with GDPR. This should be done without changing the way of working in the teams too much, as it could create a hindrance for them. A standard for workflow in all teams would both be easier to manage on a national level and would also make it possible to verify that it complies with relevant laws. As most teams seem to exclude the current CRM from their workflow, the easiest option at the moment would be to temporarily move completely to G Suite by changing the registration procedure as described in section \ref{result:google-suite}. A more long term sustainable way of handling and storing information would be to invest time (and money) into customizing and cleaning the current CRM system. There is already knowledge of this system within the organization which makes it one of the best choices for a starting point for a custom system. Other custom (or already existing CRM systems) are usually way too expensive, whilst still not perfectly satisfying the organizations needs.

% \begin{enumerate}
% \item \textbf{Google Suite}
% \newline Google, you can link it in wordpress.
% https://www.google.com/cloud/security/gdpr/ 
% Google takes responsibility for law suits? 
% \item \textbf{Clean current platform}
% \newline A cleaning up would make it better, make sure it follows the GDRP law, and invest in education- better documentation! For example make a monthly cleaning day- "tech- day", so it is not in the daily task but something everyone are contributing in monthly.
% \end{enumerate}
